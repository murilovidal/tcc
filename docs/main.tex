% Este documento destina-se a servir como modelo para a produção de documentos
% de pesquisa do PPGINF/UFPR, como projetos, dissertações e teses. A classe de
% documento se chama "ppginf" (arquivo ppginf.cls) e define o formato básico do
% documento. O texto está organizado em capítulos que são colocados em
% subdiretórios separados. São definidos exemplos para a inclusão de figuras,
% códigos-fonte e a definição de tabelas.
%
% Produzido por Carlos Maziero (maziero@inf.ufpr.br) em Outubro de 2015.
% Baseado em um modelo anterior construído pelo autor para o PPGIA/PUCPR.

% Opções da classe ppginf:
%
% - defesa  : versão para entregar à banca; tem espaçamento 1,5
%             e omite algumas páginas iniciais (agradecimentos, etc)
% - final   : versão pós-defesa, para enviar à biblioteca;
%             tem espaçamento simples e todas as páginas iniciais.
% - oneside : somente frente; use quando for gerar somente o PDF.
% - twoside : frente/verso; use quando for gerar uma versão impressa.
% - ...     : demais opções aceitas pela classe "book"

% Sobre línguas: o modelo tem suporte para português e inglês. As duas 
% línguas devem ser informadas como opção da classe; a língua principal
% do documento deve vir POR ÚLTIMO.

% Opções default: defesa, oneside, texto principal em português
\documentclass[defesa,oneside,english,brazilian]{ppginf}	% defesa
%\documentclass[defesa,oneside,brazilian,english]{ppginf}	% defesa, inglês
%\documentclass[final,oneside,english,brazilian]{ppginf}	% final PDF
%\documentclass[final,twoside,english,brazilian]{ppginf}	% final impressa

% configurações de diversos pacotes, inclusive o fonte principal do texto
\include{packages}
\usepackage[section]{placeins}

%=====================================================

\begin {document}

% Principais dados, usados para gerar as páginas iniciais.
% Campos não utilizados podem ser removidos ou comentados.

% título
\title{Sistema de Interações Sociais Baseadas em Similaridade: Uma Análise Determinística para o Cálculo do Peso das Arestas em um Grafo}

% palavras-chave e keywords (p/ resumo, abstract e metadados do PDF)
\pchave{grafo, algoritmo de recomendação, rede social, mídia social}
\keyword{graph, recommendation algorithm, social network, social media}

% autoria
\author{Carlos Felipe Amorim Paulino e Murilo Vidal}
\advisor{Prof. Dr. Alexander Kutzke}


% instituição
\iflanguage{brazilian}
  { \instit{UFPR}{Universidade Federal do Paraná} }
  { \instit{UFPR}{Federal University of Paraná} }

% área de concentração (default do PPGInf, não mudar)
\iflanguage{brazilian}
  { \field{Ciência da Computação} }
  { \field{Computer Science} }

% data (só o ano)
\date{2019}

% local
\iflanguage{brazilian}
  { \local{Curitiba PR} }
  { \local{Curitiba PR - Brazil} }

% imagem de fundo da capa (se não desejar, basta comentar)
\coverimage{0-iniciais/fundo-capa.jpg}

%=====================================================

%% Descrição do documento (obviamente, descomentar somente UMA!)

\iflanguage{brazilian}
{
% tese de doutorado
\descr{Trabalho de conclusão de curso apresentado como requisito parcial à obtenção do grau de Tecnólogo em Análise e Desenvolvimento de Sistemas, Setor de Educação Profissional e Tecnológica da Universidade Federal do Paraná}

% exame de qualificação de doutorado
%\descr{Documento apresentado como requisito parcial ao exame de qualificação de Doutorado no Programa de Pós-Graduação em Informática, Setor de Ciências Exatas, da Universidade Federal do Paraná}

% dissertação de mestrado
%\descr{Dissertação apresentada como requisito parcial à obtenção do grau de Mestre em Informática no Programa de Pós-Graduação em Informática, Setor de Ciências Exatas, da Universidade Federal do Paraná}

% exame de qualificação de mestrado
%\descr{Documento apresentado como requisito parcial ao exame de qualificação de Mestrado no Programa de Pós-Graduação em Informática, Setor de Ciências Exatas, da Universidade Federal do Paraná}

% trabalho de conclusão de curso
%\descr{Trabalho apresentado como requisito parcial à conclusão do Curso de Bacharelado em XYZ, Setor de Ciências Exatas, da Universidade Federal do Paraná}

% trabalho de disciplina
%\descr{Trabalho apresentado como requisito parcial à conclusão da disciplina XYZ no Curso de Bacharelado em XYZ, Setor de Ciências Exatas, da Universidade Federal do Paraná}
}
{
% doctorate thesis
\descr{Final paper presented as a partial requirement for the degree of Systems Analysis and Develpment Tecnologist, Professional and Tecnological Education Sector, of the Federal University of Paraná, Brazil}

% doctorate qualification
%\descr{Document presented as a partial requirement for the doctoral qualification exam in the Graduate Program in Informatics, Exact Sciences Sector, of the Federal University of Paraná, Brazil}

% MSc dissertation
%\descr{Dissertation presented as partial a requirement for the degree of Master of Sciences in Informatics in the Graduate Program in Informatics, Exact Sciences Sector, of the Federal University of Paraná, Brazil.}

% MSc qualification
%\descr{Document presented as a partial requirement for the Master of Sciences qualification exam in the Graduate Program in Informatics, Exact Sciences Sector, of the Federal University of Paraná, Brazil}

% trabalho de conclusão de curso, inglês
%\descr{to be translated}

% trabalho de disciplina, inglês
%\descr{to be translated}
}

%=====================================================

% define estilo das páginas iniciais (capas, resumo, sumário, etc)
\frontmatter
\pagestyle{frontmatter}

% define capa e folha de rosto
\titlepage

% páginas que só aparecem na versão final (a inclusão é automática)
\include{0-iniciais/catalografica}	% ficha catalográfica
\include{0-iniciais/aprovacao}		% folha de aprovação
\begin{dedica}  % só gera conteúdo se for na versão final

O aumento do conhecimento é como uma esfera dilatando-se no espaço: quanto maior a nossa compreensão, maior o nosso contato com o desconhecido.

Blaise Pascal

\end{dedica}

		% dedicatória
\begin{agradece}	% só gera conteúdo se for na versão final

Somos muitíssimo gratos a nossos pais, Josefa e José, e Clarice e Adriano, pelo apoio e carinho. Um particular agradecimento às nossas companheiras, Fernanda e Bruna, pela paciência, compreensão e, sempre disponível, atenção. Vocês foram essenciais em todos os momentos do desenvolvimento deste trabalho. Obrigado, Julia, pela disponibilidade e amizade, seu talento deu cores ao nosso sistema.

Ao professor Alex, nossos sinceros agradecimentos pela orientação, conhecimento compartilhado, incentivo e bons sentimentos. Ao professor Jouglas e ao professor Mauro, por comporem a banca de avaliação e nos iluminarem com bons comentários e críticas, nosso reconhecimento e gratidão.



\end{agradece}

		% agradecimentos

% resumo (português) e abstract (inglês)
\begin{resumo}
\singlespacing
\noindent
As relações sociais são cada vez mais realizadas em ambiente virtual, o que traz a possibilidade de encontrar pessoas de várias culturas e com diferentes opiniões. Porém, por vezes o usuário da rede social quer apenas encontrar seus pares e compartilhar interesses mútuos. Este trabalho visa construir uma rede social com foco em sugestão de novos relacionamentos baseados em interesses mútuos, sem priorizar a aparência do usuário. Nesta rede, os usuários podem relacionar-se anonimamente por uma dinâmica de perguntas e respostas, na qual é possível compartilhar questões, responder perguntas dos outros usuários e avaliar as respostas recebidas. Conforme a interação nesse modelo vai acontecendo, o sistema convida os contatos considerados similares para relacionarem-se mais proximamente pelo uso de um mensageiro instantâneo particular. Dessa maneira, os usuários entram em contato com pessoas novas tendo sido mensurada a similaridade entre cada usuário com base em um cálculo determinístico. A rede social tratada neste trabalho é representada por um grafo no qual os nós são os usuários, as arestas são as conexões entre eles e o peso das arestas determina o grau de similaridade entre os usuários. Este método de sugestão de contatos tem a ambição de ser eficaz, de maneira que demande pouca carga de processamento, pois, é uma função do número de respostas recebidas e avaliadas. Ao longo do desenvolvimento do software o cálculo do peso da aresta foi testado com várias equações e os resultados foram comparados  entre si. Uma equação foi selecionada como a mais adequada e os resultados foram considerados satisfatórios. O cálculo do valor do peso da aresta do grafo é eficiente e leve; o produto final é atrativo e a dinâmica de perguntas e respostas tem a possibilidade de direcionar o usuário para relacionamentos que tratem de assuntos que lhe interessem naturalmente. Outras possibilidades de pesquisas futuras foram iluminadas com este trabalho de maneira que surgiram novas ideias de incremento na sofisticação do cálculo do peso das arestas do grafo, seja pelo emprego de inteligência artificial ou pela mineração de dados provenientes das interações por meio da análise semântica dos textos compartilhados entre os usuários.

\end{resumo}

\begin{abstract}
\singlespacing
\noindent
The social relationships are increasingly taking place in the virtual environment, which brings the possibility of meeting people of various cultures and with divergent opinions. However, sometimes the user of the social media wishes only to meet his peers and share common interests. This work seeks to build a social media focused in suggesting new relationships based in common interests, without prioritizing the looks of the user. In this network, the users can relate anonymously by a dynamic of questions and answers, in which is possible to share questions, answer another users’ questions and evaluate the received answers. As the interaction in this model is happening, the system invites the contacts considered similar to relate closely by the use of a private instant messenger. That way, the users get in touch with new people as the similarity between each user is measured based in a deterministic calculation. The social media subject of this work is represented by a graph in which the nodes are the users, the edges are the connexion between them and the weight of the edges determines the grade of similarity between the users. This method of contacts suggestion has the ambition of being efficient, in a way that it demands little processing power as it is simply a function of the number of answers received and evaluated. Throughout the software development the edge weight calculation was tested with several equations and the results were compared to each other. An equation was selected as the most suitable and the results were considered satisfactory. The edge weight calculation is lightweight and efficient; the final product is attractive and the dynamic of questions and answers has the possibility of directing the user to relationships that  deal with subjects that interest them naturally. Another possibilities of further research were unveiled with this work as new ideas to increase the sophistication of the graph’s edge’s weight calculation, be it by employing artificial intelligence or by mining data from the interactions thru the semantic analysis of the texts shared between the users.

\end{abstract}


% listas  de figuras, tabelas, abreviações/siglas, símbolos
\listoffigures
\clearpage
\listoftables
%=====================================================

% lista de acrônimos (siglas e abreviações)

\begin{listaacron}

\begin{longtable}[l]{p{0.2\linewidth}p{0.7\linewidth}}
MVP & \emph{Minimum viable product}\\
SDK & \emph{Software development kit}\\
TADS & Tecnologia em Análise e Desenvolvimento de Sistemas\\
UFPR & Universidade Federal do Paraná\\
VM & \emph{Virtual Machine}\\
VPS & \emph{Virtual Private Server}\\
\end{longtable}

\end{listaacron}

%=====================================================
		% ainda deve ser preenchida à mão
%=====================================================

% lista de símbolos

\begin{listasimb}

\begin{longtable}[l]{p{0.2\linewidth}p{0.7\linewidth}}
$\psi$ & psi, vigésima terceira letra do alfabeto grego\\
\end{longtable}

\end{listasimb}

%=====================================================
		% idem

% sumário
\tableofcontents

%=====================================================

% define estilo do corpo do documento (capítulos e apêndices)
\mainmatter
\pagestyle{mainmatter}

% inclusao de cada capítulo, alterar a gosto (do professor de Metodologia)
\chapter{Metodologia}
\label{cap:metodologia}

% figuras estão no subdiretório "figuras/" dentro deste capítulo
\graphicspath{\currfiledir/figuras/}

%=====================================================

\section{Método de desenvolvimento}

O produto foi desenvolvido pelo modelo de cascata. A priorização das atividades foi baseada na competência da equipe em cada área de desenvolvimento. Dessa maneira, o \emph{front end} foi considerado o foco e a prioridade no desenvolvimento, tendo em vista que o nível de conhecimento da equipe na área de design e usabilidade era perceptivelmente menor do que a habilidade para o desenvolvimento do \emph{back end}.

A análise de requisitos…

Um MVP, foi desenvolvido durante a fase de projeto. A figura \ref{fig:mvpCriarQuest} apresenta a tela de criação de perguntas do MVP. Neste produto, é possível criar perguntas (\ref{fig:mvpCriarQuest}), responder perguntas postadas por outros usuários (\ref{fig:mvpVerQuest}) e marcar essas respostas (\ref{fig:mvpVerResp}). Assim, o produto já tem informações suficientes para criar um grafo com os usuários no qual o peso das arestas é o nível de afinidade entre os eles baseado na quantidade de respostas marcadas.

\begin{figure}[!htb]
\centering
\includegraphics[width=14cm]{mvpCriarQuest.png}
\caption{Tela de criação de perguntas do MVP. Fonte: os autores.}
\label{fig:mvpCriarQuest}
\end{figure}

\begin{figure}[!htb]
\centering
\includegraphics[width=14cm]{mvpVerQuest.png}
\caption{Tela de visualização de perguntas do MVP. Nesta tela é possível escolher uma pergunta para ser respondida. Fonte: os autores.}
\label{fig:mvpVerQuest}
\end{figure}

\begin{figure}[!htb]
\centering
\includegraphics[width=14cm]{mvpVerResp.png}
\caption{Tela de visualização de respostas recebidas do MVP. Nesta tela é possível marcar perguntas favoritas. Fonte: os autores.}
\label{fig:mvpVerResp}
\end{figure}

\begin{figure}[!htb]
\centering
\includegraphics[width=14cm]{mvpVerGrafo.png}
\caption{Visualização do grafo que representa as ligações entre usuários da rede social no MVP. Fonte: os autores.}
\label{fig:mvpVerGrafo}
\end{figure}

\FloatBarrier


A figura \ref{fig:mvpVerGrafo} é a tela do MVP que representa os usuários e suas conexões criadas na rede social por meio de um grafo.


\section{Arquitetura}

\begin{itemize}
\item Qual a tecnologia
\item Custo
\item Arquitetura
\end{itemize}

Desenho da arquitetura

\begin{figure}[!htb]
\centering
\includegraphics[width=14cm]{arquitetura.png}
\caption{Arquitetura do software. Fonte: os autores.}
\label{fig:arquitetura}
\end{figure}

\FloatBarrier
%=====================================================

\section{Tecnologia aplicada}

\begin{itemize}
\item Linguagens de programação
\item Frameworks

\end{itemize}

Frameworks: IONIC django

angular e github.

Linguagens: python typescript javascript html/css
Hardware: PC Notebook thinkpad x201, SSD 120GB e 8GB de RAM primeira geracao do i5 e arch linux, server usa debian stretch
Infraestrutura: 	Docker
VPS: CPU: Intel (Haswell, noTSX) (1)@2.3GHz GPU: CirrusLogicGD5446 Memory: 583MB / 1956MB.



%=====================================================

\section{Análise do sistema}

Diagramas de caso de USO
\begin{figure}[!htb]
\centering
\includegraphics[width=16cm]{DCU1.png}
\caption{Diagrama de caso de uso nível 1. Fonte: os autores.}
\label{fig:DCU1}
\end{figure}

\begin{figure}[!htb]
\centering
\includegraphics[width=16cm]{DCU2.png}
\caption{Diagrama de caso de uso nível 2. Fonte: os autores.}
\label{fig:DCU2}
\end{figure}
%=====================================================

Diagramas de sequência

\begin{figure}[!htb]
\centering
\includegraphics[width=16cm]{UC001-VisualizarQuestao.png}
\caption{Diagrama de caso de uso UC001 - Visualizar Questão. Fonte: os autores.}
\label{fig:UC001}
\end{figure}

\begin{figure}[!htb]
\centering
\includegraphics[width=16cm]{UC002-ResponderQuestao.png}
\caption{Diagrama de caso de uso UC002 - Responder Questão. Fonte: os autores.}
\label{fig:UC002}
\end{figure}

\begin{figure}[!htb]
\centering
\includegraphics[width=16cm]{UC003-VisualizarRespostas.png}
\caption{Diagrama de caso de uso UC003 - Visualizar Respostas. Fonte: os autores.}
\label{fig:UC003}
\end{figure}


\begin{figure}[!htb]
\centering
\includegraphics[width=16cm]{UC004-CriarQuestao.png}
\caption{Diagrama de caso de uso UC004 - Criar Questão. Fonte: os autores.}
\label{fig:UC004}
\end{figure}


\begin{figure}[!htb]
\centering
\includegraphics[width=16cm]{UC005-VisualizarCombinacoes.png}
\caption{Diagrama de caso de uso UC005 - Visualizar Combinações. Fonte: os autores.}
\label{fig:UC005}
\end{figure}


Diagrama de Classes

\begin{figure}[!htb]
\centering
\includegraphics[width=16cm]{DiagramaClasse.png}
\caption{Diagrama de classes. Fonte: os autores.}
\label{fig:diagramaClasse}
\end{figure}
\FloatBarrier



%=====================================================

\section{Cálculo de do peso da aresta}
O peso de uma aresta do grafo, $k$, representa o nível de afinidade entre os dois usuários conectados por esta aresta. Um usuário pode estar conectado a vários outros usuários.

Quando um usuário entra pela primeira vez na rede social, ele é obrigado a preencher um questionário contendo as questões constantes na tabela \ref{tab:questoes}. A partir das respostas deste questionário, um valor para $k$ é calculado levando em conta, tão somente, a similaridade entre as respostas de cada usuário.

\begin{table}[!htp]
\centering
\caption{Formulário inicial}
\label{tab:questoes}
\begin{tabular}{ || c ||}
\hline
Pergunta\\
\hline
\hline
Você prefere cachorro ou gato?\\  
\hline
Você prefere rock ou funk?\\
\hline
Você prefere verão ou inverno? \\
\hline
Você prefere cinema ou teatro?\\
\hline
Você prefere cerveja ou vinho?\\
\hline
Você prefere o dia ou a noite?\\
\hline
Você prefere sair ou ficar em casa?\\
\hline
Você fuma?\\
\hline
Você tem alguma religião?\\
\hline
Você acredita em signos?\\
\hline
Você prefere praia ou campo?\\
\hline
\end{tabular}
\end{table}

Para a determinação inicial de $k$, logo após o preenchimento do formulário, é calculado o total de respostas iguais entre dois usuários e aplicada a seguinte equação:

\begin{equation}
k_{ab} = \frac{R_{ab}*(x-1)}{N_{p}}
 \label{eq:k0}
\end{equation}


Onde $R_{ab}$ é o total de respostas do usuário A iguais ao usuário B; $x$ é o valor objetivo para considerar dois usuários similares - este valor será discutido adiante nesta seção - e $N_{p}$ é o total de perguntas do questionário inicial.

Na equação \ref{eq:k0}, o numerador é multiplicado por $x-1$ para que os usuários que tiveram todas as questões respondidas da mesma maneira no questionário tão somente fiquem muito próximos da margem que define a habilitação do mensageiro instantâneo. O objetivo é tornar obrigatória a interação por meio de perguntas e respostas antes que dois usuários possam ser considerados similares o suficiente para desfrutarem do mensageiro.

Então, conforme as perguntas postadas pelos usuários vão sendo respondidas e apreciadas, valor do peso da aresta, definido como $k_{ab}$, que liga os usuários $A$ e $B$, é recalculado, para cada resposta apontada como apreciada, a partir da seguinte definição:

\begin{equation}
k_{ab} = k_{ab} + \frac{(R_{ab} + R_{ba})}{(P_{A} + P_{B})}
\label{eq:k1}
\end{equation}

Onde $R_{ab}$ é o número de respostas que o usuário $A$ recebeu do usuário $B$ e $A$ gostou; $R_{ba}$ é o número de respostas que o usuário $B$ recebeu do usuário $A$ e a $B$ gostou; $P_{A}$ é o número de perguntas postadas pelo usuário $A$ e $P_{B}$ o número de respostas postadas pelo usuário $B$.

Na figura \ref{fig:grafico_k1}, podemos ver a evolução do peso $k$ para cada resposta recebida pelo usuário $A$ que ele gostou. Nesta situação o crescimento do valor de $k$ é linear, pois todas as perguntas postadas pelo usuário $A$ estão respondidas por $B$ e as respectivas respostas foram curtidas.

\begin{figure}[!htb]
\centering
\includegraphics[width=14cm]{grafico_k1.png}
\caption{Evolução de $k$ se todas as perguntas postadas por $A$ forem respondidas por $B$. Fonte: os autores.}
\label{fig:grafico_k1}
\end{figure}

Como o cálculo leva em conta todas as perguntas postadas pelo usuário $A$, no caso de já haver perguntas postadas e não respondidas, o crescimento do valor de $k$ é afetado num grau inversamente proporcional ao número de perguntas postadas, como pode-se observar na figura \ref{fig:grafico_k2}.

\begin{figure}[!htb]
\centering
\includegraphics[width=14cm]{grafico_k2.png}
\caption{Evolução de $k$ se $A$ postou 50 perguntas e apenas 25 respostas de $B$ foram curtidas. Fonte: os autores.}
\label{fig:grafico_k2}
\end{figure}

É importante observar que o peso do incremento para $k$ é menor quando maior for o número de perguntas postadas seja por $A$ ou $B$. Dessa maneira, podemos depreender que a interação entre usuários novos, ou com poucas perguntas postadas, terá influência maior no crescimento do valor de $k$. Por outro lado, a interação entre usuários com muitas perguntas já postadas, incrementará um valor menor sobre $k$.

Tendo calculado um peso para cada aresta que liga os usuários, foi necessário definir um limiar $x$ que seria o valor mínimo de $k$ para considerar que dois usuários são similares o suficiente para serem postos em contato sob o mensageiro instantâneo.

A definição de $x$ provou-se um desafio, tendo em vista que o comportamento de $k$ varia em função do número total de perguntas postadas por dois usuários em análise. A figura \ref{fig:poucasquestoes} mostra que a convergência para $x = 1$ para o caso de de dois usuários que postaram poucas perguntas é rápida, pois o incremento de $k$ é maior. No caso simulado na figura \ref{fig:poucasquestões}, pouco mais de cinco respostas curtidas foram o suficiente para atingir o $x$.

\begin{figure}[!htb]
\centering
\includegraphics[width=14cm]{poucasquestoes.png}
\caption{Evolução de $k$ em função das respostas curtidas se $A$ e $B$ têm poucas perguntas postadas. Ex.: menos que 10 perguntas. Fonte: os autores.}
\label{fig:poucasquestoes}
\end{figure}

A figura \ref{fig:muitasquestoes}  demonstra que a convergência de $k$ e $x$ para usuários que já postaram muitas perguntas demanda muito mais respostas curtidas, neste caso mais que vinte respostas. Isto deve-se ao fato de ter sido decidido que a equação \ref{eq:k1} tem como divisor a soma de todas as perguntas já postadas pelos dois usuários, portanto, o incremento de $k$ é inversamente proporcional à soma de todas as perguntas postadas pelos usuários. 

\begin{figure}[!htb]
\centering
\includegraphics[width=14cm]{muitasquestoes.png}
\caption{Evolução de $k$ em função das respostas curtidas se $A$ e $B$ têm muitas perguntas postadas. Ex.: mais que 50 perguntas. Fonte: os autores.}
\label{fig:muitasquestoes}
\end{figure}

Uma adequação simplista para este problema seria definir $k$ como o número total de respostas curtidas entre dois usuários. Assim, $x$ seria o número mínimo de respostas que teriam que ser curtidas entre os usuários para que fossem considerados similares o suficiente para contatarem-se pelo mensageiro instantâneo. Porém, dessa maneira ignora-se que, se houverem muitas perguntas postadas por $A$ sem resposta de $B$ é possível que hajam mais diferenças do que similaridades entre os usuários, haja vista que muitas perguntas são ignoradas.

Portanto, é imprescindível considerar o volume de perguntas postadas. Uma evolução poderia adequar o software para contabilizar o número de perguntas postadas por $A$ e \emph{vistas} por $B$. Assim, poderíamos adequar o cálculo de $k$ como segue: 

\begin{equation}
k = k + \frac{R_{ab} + R_{ba}}{P_{AvB} + P_{BvA}} 
\label{eq:k2}
\end{equation}

Onde $R_{ab}$ é o número de respostas que o usuário $A$ recebeu do usuário $B$ e $A$ gostou; $R_{ba}$ é o número de respostas que o usuário $B$ recebeu do usuário $A$ e a $B$ gostou; $P_{AvB}$ é o número de perguntas postadas pelo usuário $A$ e vistas por $B$ e $P_{BvA}$ o número de respostas postadas pelo usuário $B$ e vistas por $A$.

Aplicando a equação \ref{eq:k2}, o incremento do valor de $k$ para cada resposta curtida seria menor para o caso de muitas perguntas terem sido vistas e poucas respostas terem sido dadas. De maneira inversa, quanto menor o conjunto de perguntas trocadas entre os usuários, maior o peso que uma resposta curtida terá sobre $k$.

Seria conveniente também, afetar as perguntas postadas com uma validade por cronologia ou fixar um valor máximo para o total de perguntas postadas pelos usuários $A$ e $B$. Para a implementação do produto neste trabalho, não foi considerada a idade das perguntas nem estabelecido um valor limite do somatório de perguntas postadas por $A$ e $B$ pois seria inconveniente para o desenvolvimento.

%=====================================================



			% introdução
\chapter{Metodologia}
\label{cap:metodologia}

% figuras estão no subdiretório "figuras/" dentro deste capítulo
\graphicspath{\currfiledir/figuras/}

%=====================================================

\section{Método de desenvolvimento}

O produto foi desenvolvido pelo modelo de cascata. A priorização das atividades foi baseada na competência da equipe em cada área de desenvolvimento. Dessa maneira, o \emph{front end} foi considerado o foco e a prioridade no desenvolvimento, tendo em vista que o nível de conhecimento da equipe na área de design e usabilidade era perceptivelmente menor do que a habilidade para o desenvolvimento do \emph{back end}.

A análise de requisitos…

Um MVP, foi desenvolvido durante a fase de projeto. A figura \ref{fig:mvpCriarQuest} apresenta a tela de criação de perguntas do MVP. Neste produto, é possível criar perguntas (\ref{fig:mvpCriarQuest}), responder perguntas postadas por outros usuários (\ref{fig:mvpVerQuest}) e marcar essas respostas (\ref{fig:mvpVerResp}). Assim, o produto já tem informações suficientes para criar um grafo com os usuários no qual o peso das arestas é o nível de afinidade entre os eles baseado na quantidade de respostas marcadas.

\begin{figure}[!htb]
\centering
\includegraphics[width=14cm]{mvpCriarQuest.png}
\caption{Tela de criação de perguntas do MVP. Fonte: os autores.}
\label{fig:mvpCriarQuest}
\end{figure}

\begin{figure}[!htb]
\centering
\includegraphics[width=14cm]{mvpVerQuest.png}
\caption{Tela de visualização de perguntas do MVP. Nesta tela é possível escolher uma pergunta para ser respondida. Fonte: os autores.}
\label{fig:mvpVerQuest}
\end{figure}

\begin{figure}[!htb]
\centering
\includegraphics[width=14cm]{mvpVerResp.png}
\caption{Tela de visualização de respostas recebidas do MVP. Nesta tela é possível marcar perguntas favoritas. Fonte: os autores.}
\label{fig:mvpVerResp}
\end{figure}

\begin{figure}[!htb]
\centering
\includegraphics[width=14cm]{mvpVerGrafo.png}
\caption{Visualização do grafo que representa as ligações entre usuários da rede social no MVP. Fonte: os autores.}
\label{fig:mvpVerGrafo}
\end{figure}

\FloatBarrier


A figura \ref{fig:mvpVerGrafo} é a tela do MVP que representa os usuários e suas conexões criadas na rede social por meio de um grafo.


\section{Arquitetura}

\begin{itemize}
\item Qual a tecnologia
\item Custo
\item Arquitetura
\end{itemize}

Desenho da arquitetura

\begin{figure}[!htb]
\centering
\includegraphics[width=14cm]{arquitetura.png}
\caption{Arquitetura do software. Fonte: os autores.}
\label{fig:arquitetura}
\end{figure}

\FloatBarrier
%=====================================================

\section{Tecnologia aplicada}

\begin{itemize}
\item Linguagens de programação
\item Frameworks

\end{itemize}

Frameworks: IONIC django

angular e github.

Linguagens: python typescript javascript html/css
Hardware: PC Notebook thinkpad x201, SSD 120GB e 8GB de RAM primeira geracao do i5 e arch linux, server usa debian stretch
Infraestrutura: 	Docker
VPS: CPU: Intel (Haswell, noTSX) (1)@2.3GHz GPU: CirrusLogicGD5446 Memory: 583MB / 1956MB.



%=====================================================

\section{Análise do sistema}

Diagramas de caso de USO
\begin{figure}[!htb]
\centering
\includegraphics[width=16cm]{DCU1.png}
\caption{Diagrama de caso de uso nível 1. Fonte: os autores.}
\label{fig:DCU1}
\end{figure}

\begin{figure}[!htb]
\centering
\includegraphics[width=16cm]{DCU2.png}
\caption{Diagrama de caso de uso nível 2. Fonte: os autores.}
\label{fig:DCU2}
\end{figure}
%=====================================================

Diagramas de sequência

\begin{figure}[!htb]
\centering
\includegraphics[width=16cm]{UC001-VisualizarQuestao.png}
\caption{Diagrama de caso de uso UC001 - Visualizar Questão. Fonte: os autores.}
\label{fig:UC001}
\end{figure}

\begin{figure}[!htb]
\centering
\includegraphics[width=16cm]{UC002-ResponderQuestao.png}
\caption{Diagrama de caso de uso UC002 - Responder Questão. Fonte: os autores.}
\label{fig:UC002}
\end{figure}

\begin{figure}[!htb]
\centering
\includegraphics[width=16cm]{UC003-VisualizarRespostas.png}
\caption{Diagrama de caso de uso UC003 - Visualizar Respostas. Fonte: os autores.}
\label{fig:UC003}
\end{figure}


\begin{figure}[!htb]
\centering
\includegraphics[width=16cm]{UC004-CriarQuestao.png}
\caption{Diagrama de caso de uso UC004 - Criar Questão. Fonte: os autores.}
\label{fig:UC004}
\end{figure}


\begin{figure}[!htb]
\centering
\includegraphics[width=16cm]{UC005-VisualizarCombinacoes.png}
\caption{Diagrama de caso de uso UC005 - Visualizar Combinações. Fonte: os autores.}
\label{fig:UC005}
\end{figure}


Diagrama de Classes

\begin{figure}[!htb]
\centering
\includegraphics[width=16cm]{DiagramaClasse.png}
\caption{Diagrama de classes. Fonte: os autores.}
\label{fig:diagramaClasse}
\end{figure}
\FloatBarrier



%=====================================================

\section{Cálculo de do peso da aresta}
O peso de uma aresta do grafo, $k$, representa o nível de afinidade entre os dois usuários conectados por esta aresta. Um usuário pode estar conectado a vários outros usuários.

Quando um usuário entra pela primeira vez na rede social, ele é obrigado a preencher um questionário contendo as questões constantes na tabela \ref{tab:questoes}. A partir das respostas deste questionário, um valor para $k$ é calculado levando em conta, tão somente, a similaridade entre as respostas de cada usuário.

\begin{table}[!htp]
\centering
\caption{Formulário inicial}
\label{tab:questoes}
\begin{tabular}{ || c ||}
\hline
Pergunta\\
\hline
\hline
Você prefere cachorro ou gato?\\  
\hline
Você prefere rock ou funk?\\
\hline
Você prefere verão ou inverno? \\
\hline
Você prefere cinema ou teatro?\\
\hline
Você prefere cerveja ou vinho?\\
\hline
Você prefere o dia ou a noite?\\
\hline
Você prefere sair ou ficar em casa?\\
\hline
Você fuma?\\
\hline
Você tem alguma religião?\\
\hline
Você acredita em signos?\\
\hline
Você prefere praia ou campo?\\
\hline
\end{tabular}
\end{table}

Para a determinação inicial de $k$, logo após o preenchimento do formulário, é calculado o total de respostas iguais entre dois usuários e aplicada a seguinte equação:

\begin{equation}
k_{ab} = \frac{R_{ab}*(x-1)}{N_{p}}
 \label{eq:k0}
\end{equation}


Onde $R_{ab}$ é o total de respostas do usuário A iguais ao usuário B; $x$ é o valor objetivo para considerar dois usuários similares - este valor será discutido adiante nesta seção - e $N_{p}$ é o total de perguntas do questionário inicial.

Na equação \ref{eq:k0}, o numerador é multiplicado por $x-1$ para que os usuários que tiveram todas as questões respondidas da mesma maneira no questionário tão somente fiquem muito próximos da margem que define a habilitação do mensageiro instantâneo. O objetivo é tornar obrigatória a interação por meio de perguntas e respostas antes que dois usuários possam ser considerados similares o suficiente para desfrutarem do mensageiro.

Então, conforme as perguntas postadas pelos usuários vão sendo respondidas e apreciadas, valor do peso da aresta, definido como $k_{ab}$, que liga os usuários $A$ e $B$, é recalculado, para cada resposta apontada como apreciada, a partir da seguinte definição:

\begin{equation}
k_{ab} = k_{ab} + \frac{(R_{ab} + R_{ba})}{(P_{A} + P_{B})}
\label{eq:k1}
\end{equation}

Onde $R_{ab}$ é o número de respostas que o usuário $A$ recebeu do usuário $B$ e $A$ gostou; $R_{ba}$ é o número de respostas que o usuário $B$ recebeu do usuário $A$ e a $B$ gostou; $P_{A}$ é o número de perguntas postadas pelo usuário $A$ e $P_{B}$ o número de respostas postadas pelo usuário $B$.

Na figura \ref{fig:grafico_k1}, podemos ver a evolução do peso $k$ para cada resposta recebida pelo usuário $A$ que ele gostou. Nesta situação o crescimento do valor de $k$ é linear, pois todas as perguntas postadas pelo usuário $A$ estão respondidas por $B$ e as respectivas respostas foram curtidas.

\begin{figure}[!htb]
\centering
\includegraphics[width=14cm]{grafico_k1.png}
\caption{Evolução de $k$ se todas as perguntas postadas por $A$ forem respondidas por $B$. Fonte: os autores.}
\label{fig:grafico_k1}
\end{figure}

Como o cálculo leva em conta todas as perguntas postadas pelo usuário $A$, no caso de já haver perguntas postadas e não respondidas, o crescimento do valor de $k$ é afetado num grau inversamente proporcional ao número de perguntas postadas, como pode-se observar na figura \ref{fig:grafico_k2}.

\begin{figure}[!htb]
\centering
\includegraphics[width=14cm]{grafico_k2.png}
\caption{Evolução de $k$ se $A$ postou 50 perguntas e apenas 25 respostas de $B$ foram curtidas. Fonte: os autores.}
\label{fig:grafico_k2}
\end{figure}

É importante observar que o peso do incremento para $k$ é menor quando maior for o número de perguntas postadas seja por $A$ ou $B$. Dessa maneira, podemos depreender que a interação entre usuários novos, ou com poucas perguntas postadas, terá influência maior no crescimento do valor de $k$. Por outro lado, a interação entre usuários com muitas perguntas já postadas, incrementará um valor menor sobre $k$.

Tendo calculado um peso para cada aresta que liga os usuários, foi necessário definir um limiar $x$ que seria o valor mínimo de $k$ para considerar que dois usuários são similares o suficiente para serem postos em contato sob o mensageiro instantâneo.

A definição de $x$ provou-se um desafio, tendo em vista que o comportamento de $k$ varia em função do número total de perguntas postadas por dois usuários em análise. A figura \ref{fig:poucasquestoes} mostra que a convergência para $x = 1$ para o caso de de dois usuários que postaram poucas perguntas é rápida, pois o incremento de $k$ é maior. No caso simulado na figura \ref{fig:poucasquestões}, pouco mais de cinco respostas curtidas foram o suficiente para atingir o $x$.

\begin{figure}[!htb]
\centering
\includegraphics[width=14cm]{poucasquestoes.png}
\caption{Evolução de $k$ em função das respostas curtidas se $A$ e $B$ têm poucas perguntas postadas. Ex.: menos que 10 perguntas. Fonte: os autores.}
\label{fig:poucasquestoes}
\end{figure}

A figura \ref{fig:muitasquestoes}  demonstra que a convergência de $k$ e $x$ para usuários que já postaram muitas perguntas demanda muito mais respostas curtidas, neste caso mais que vinte respostas. Isto deve-se ao fato de ter sido decidido que a equação \ref{eq:k1} tem como divisor a soma de todas as perguntas já postadas pelos dois usuários, portanto, o incremento de $k$ é inversamente proporcional à soma de todas as perguntas postadas pelos usuários. 

\begin{figure}[!htb]
\centering
\includegraphics[width=14cm]{muitasquestoes.png}
\caption{Evolução de $k$ em função das respostas curtidas se $A$ e $B$ têm muitas perguntas postadas. Ex.: mais que 50 perguntas. Fonte: os autores.}
\label{fig:muitasquestoes}
\end{figure}

Uma adequação simplista para este problema seria definir $k$ como o número total de respostas curtidas entre dois usuários. Assim, $x$ seria o número mínimo de respostas que teriam que ser curtidas entre os usuários para que fossem considerados similares o suficiente para contatarem-se pelo mensageiro instantâneo. Porém, dessa maneira ignora-se que, se houverem muitas perguntas postadas por $A$ sem resposta de $B$ é possível que hajam mais diferenças do que similaridades entre os usuários, haja vista que muitas perguntas são ignoradas.

Portanto, é imprescindível considerar o volume de perguntas postadas. Uma evolução poderia adequar o software para contabilizar o número de perguntas postadas por $A$ e \emph{vistas} por $B$. Assim, poderíamos adequar o cálculo de $k$ como segue: 

\begin{equation}
k = k + \frac{R_{ab} + R_{ba}}{P_{AvB} + P_{BvA}} 
\label{eq:k2}
\end{equation}

Onde $R_{ab}$ é o número de respostas que o usuário $A$ recebeu do usuário $B$ e $A$ gostou; $R_{ba}$ é o número de respostas que o usuário $B$ recebeu do usuário $A$ e a $B$ gostou; $P_{AvB}$ é o número de perguntas postadas pelo usuário $A$ e vistas por $B$ e $P_{BvA}$ o número de respostas postadas pelo usuário $B$ e vistas por $A$.

Aplicando a equação \ref{eq:k2}, o incremento do valor de $k$ para cada resposta curtida seria menor para o caso de muitas perguntas terem sido vistas e poucas respostas terem sido dadas. De maneira inversa, quanto menor o conjunto de perguntas trocadas entre os usuários, maior o peso que uma resposta curtida terá sobre $k$.

Seria conveniente também, afetar as perguntas postadas com uma validade por cronologia ou fixar um valor máximo para o total de perguntas postadas pelos usuários $A$ e $B$. Para a implementação do produto neste trabalho, não foi considerada a idade das perguntas nem estabelecido um valor limite do somatório de perguntas postadas por $A$ e $B$ pois seria inconveniente para o desenvolvimento.

%=====================================================



		% fundamentação teórica
\chapter{Metodologia}
\label{cap:metodologia}

% figuras estão no subdiretório "figuras/" dentro deste capítulo
\graphicspath{\currfiledir/figuras/}

%=====================================================

\section{Método de desenvolvimento}

O produto foi desenvolvido pelo modelo de cascata. A priorização das atividades foi baseada na competência da equipe em cada área de desenvolvimento. Dessa maneira, o \emph{front end} foi considerado o foco e a prioridade no desenvolvimento, tendo em vista que o nível de conhecimento da equipe na área de design e usabilidade era perceptivelmente menor do que a habilidade para o desenvolvimento do \emph{back end}.

A análise de requisitos…

Um MVP, foi desenvolvido durante a fase de projeto. A figura \ref{fig:mvpCriarQuest} apresenta a tela de criação de perguntas do MVP. Neste produto, é possível criar perguntas (\ref{fig:mvpCriarQuest}), responder perguntas postadas por outros usuários (\ref{fig:mvpVerQuest}) e marcar essas respostas (\ref{fig:mvpVerResp}). Assim, o produto já tem informações suficientes para criar um grafo com os usuários no qual o peso das arestas é o nível de afinidade entre os eles baseado na quantidade de respostas marcadas.

\begin{figure}[!htb]
\centering
\includegraphics[width=14cm]{mvpCriarQuest.png}
\caption{Tela de criação de perguntas do MVP. Fonte: os autores.}
\label{fig:mvpCriarQuest}
\end{figure}

\begin{figure}[!htb]
\centering
\includegraphics[width=14cm]{mvpVerQuest.png}
\caption{Tela de visualização de perguntas do MVP. Nesta tela é possível escolher uma pergunta para ser respondida. Fonte: os autores.}
\label{fig:mvpVerQuest}
\end{figure}

\begin{figure}[!htb]
\centering
\includegraphics[width=14cm]{mvpVerResp.png}
\caption{Tela de visualização de respostas recebidas do MVP. Nesta tela é possível marcar perguntas favoritas. Fonte: os autores.}
\label{fig:mvpVerResp}
\end{figure}

\begin{figure}[!htb]
\centering
\includegraphics[width=14cm]{mvpVerGrafo.png}
\caption{Visualização do grafo que representa as ligações entre usuários da rede social no MVP. Fonte: os autores.}
\label{fig:mvpVerGrafo}
\end{figure}

\FloatBarrier


A figura \ref{fig:mvpVerGrafo} é a tela do MVP que representa os usuários e suas conexões criadas na rede social por meio de um grafo.


\section{Arquitetura}

\begin{itemize}
\item Qual a tecnologia
\item Custo
\item Arquitetura
\end{itemize}

Desenho da arquitetura

\begin{figure}[!htb]
\centering
\includegraphics[width=14cm]{arquitetura.png}
\caption{Arquitetura do software. Fonte: os autores.}
\label{fig:arquitetura}
\end{figure}

\FloatBarrier
%=====================================================

\section{Tecnologia aplicada}

\begin{itemize}
\item Linguagens de programação
\item Frameworks

\end{itemize}

Frameworks: IONIC django

angular e github.

Linguagens: python typescript javascript html/css
Hardware: PC Notebook thinkpad x201, SSD 120GB e 8GB de RAM primeira geracao do i5 e arch linux, server usa debian stretch
Infraestrutura: 	Docker
VPS: CPU: Intel (Haswell, noTSX) (1)@2.3GHz GPU: CirrusLogicGD5446 Memory: 583MB / 1956MB.



%=====================================================

\section{Análise do sistema}

Diagramas de caso de USO
\begin{figure}[!htb]
\centering
\includegraphics[width=16cm]{DCU1.png}
\caption{Diagrama de caso de uso nível 1. Fonte: os autores.}
\label{fig:DCU1}
\end{figure}

\begin{figure}[!htb]
\centering
\includegraphics[width=16cm]{DCU2.png}
\caption{Diagrama de caso de uso nível 2. Fonte: os autores.}
\label{fig:DCU2}
\end{figure}
%=====================================================

Diagramas de sequência

\begin{figure}[!htb]
\centering
\includegraphics[width=16cm]{UC001-VisualizarQuestao.png}
\caption{Diagrama de caso de uso UC001 - Visualizar Questão. Fonte: os autores.}
\label{fig:UC001}
\end{figure}

\begin{figure}[!htb]
\centering
\includegraphics[width=16cm]{UC002-ResponderQuestao.png}
\caption{Diagrama de caso de uso UC002 - Responder Questão. Fonte: os autores.}
\label{fig:UC002}
\end{figure}

\begin{figure}[!htb]
\centering
\includegraphics[width=16cm]{UC003-VisualizarRespostas.png}
\caption{Diagrama de caso de uso UC003 - Visualizar Respostas. Fonte: os autores.}
\label{fig:UC003}
\end{figure}


\begin{figure}[!htb]
\centering
\includegraphics[width=16cm]{UC004-CriarQuestao.png}
\caption{Diagrama de caso de uso UC004 - Criar Questão. Fonte: os autores.}
\label{fig:UC004}
\end{figure}


\begin{figure}[!htb]
\centering
\includegraphics[width=16cm]{UC005-VisualizarCombinacoes.png}
\caption{Diagrama de caso de uso UC005 - Visualizar Combinações. Fonte: os autores.}
\label{fig:UC005}
\end{figure}


Diagrama de Classes

\begin{figure}[!htb]
\centering
\includegraphics[width=16cm]{DiagramaClasse.png}
\caption{Diagrama de classes. Fonte: os autores.}
\label{fig:diagramaClasse}
\end{figure}
\FloatBarrier



%=====================================================

\section{Cálculo de do peso da aresta}
O peso de uma aresta do grafo, $k$, representa o nível de afinidade entre os dois usuários conectados por esta aresta. Um usuário pode estar conectado a vários outros usuários.

Quando um usuário entra pela primeira vez na rede social, ele é obrigado a preencher um questionário contendo as questões constantes na tabela \ref{tab:questoes}. A partir das respostas deste questionário, um valor para $k$ é calculado levando em conta, tão somente, a similaridade entre as respostas de cada usuário.

\begin{table}[!htp]
\centering
\caption{Formulário inicial}
\label{tab:questoes}
\begin{tabular}{ || c ||}
\hline
Pergunta\\
\hline
\hline
Você prefere cachorro ou gato?\\  
\hline
Você prefere rock ou funk?\\
\hline
Você prefere verão ou inverno? \\
\hline
Você prefere cinema ou teatro?\\
\hline
Você prefere cerveja ou vinho?\\
\hline
Você prefere o dia ou a noite?\\
\hline
Você prefere sair ou ficar em casa?\\
\hline
Você fuma?\\
\hline
Você tem alguma religião?\\
\hline
Você acredita em signos?\\
\hline
Você prefere praia ou campo?\\
\hline
\end{tabular}
\end{table}

Para a determinação inicial de $k$, logo após o preenchimento do formulário, é calculado o total de respostas iguais entre dois usuários e aplicada a seguinte equação:

\begin{equation}
k_{ab} = \frac{R_{ab}*(x-1)}{N_{p}}
 \label{eq:k0}
\end{equation}


Onde $R_{ab}$ é o total de respostas do usuário A iguais ao usuário B; $x$ é o valor objetivo para considerar dois usuários similares - este valor será discutido adiante nesta seção - e $N_{p}$ é o total de perguntas do questionário inicial.

Na equação \ref{eq:k0}, o numerador é multiplicado por $x-1$ para que os usuários que tiveram todas as questões respondidas da mesma maneira no questionário tão somente fiquem muito próximos da margem que define a habilitação do mensageiro instantâneo. O objetivo é tornar obrigatória a interação por meio de perguntas e respostas antes que dois usuários possam ser considerados similares o suficiente para desfrutarem do mensageiro.

Então, conforme as perguntas postadas pelos usuários vão sendo respondidas e apreciadas, valor do peso da aresta, definido como $k_{ab}$, que liga os usuários $A$ e $B$, é recalculado, para cada resposta apontada como apreciada, a partir da seguinte definição:

\begin{equation}
k_{ab} = k_{ab} + \frac{(R_{ab} + R_{ba})}{(P_{A} + P_{B})}
\label{eq:k1}
\end{equation}

Onde $R_{ab}$ é o número de respostas que o usuário $A$ recebeu do usuário $B$ e $A$ gostou; $R_{ba}$ é o número de respostas que o usuário $B$ recebeu do usuário $A$ e a $B$ gostou; $P_{A}$ é o número de perguntas postadas pelo usuário $A$ e $P_{B}$ o número de respostas postadas pelo usuário $B$.

Na figura \ref{fig:grafico_k1}, podemos ver a evolução do peso $k$ para cada resposta recebida pelo usuário $A$ que ele gostou. Nesta situação o crescimento do valor de $k$ é linear, pois todas as perguntas postadas pelo usuário $A$ estão respondidas por $B$ e as respectivas respostas foram curtidas.

\begin{figure}[!htb]
\centering
\includegraphics[width=14cm]{grafico_k1.png}
\caption{Evolução de $k$ se todas as perguntas postadas por $A$ forem respondidas por $B$. Fonte: os autores.}
\label{fig:grafico_k1}
\end{figure}

Como o cálculo leva em conta todas as perguntas postadas pelo usuário $A$, no caso de já haver perguntas postadas e não respondidas, o crescimento do valor de $k$ é afetado num grau inversamente proporcional ao número de perguntas postadas, como pode-se observar na figura \ref{fig:grafico_k2}.

\begin{figure}[!htb]
\centering
\includegraphics[width=14cm]{grafico_k2.png}
\caption{Evolução de $k$ se $A$ postou 50 perguntas e apenas 25 respostas de $B$ foram curtidas. Fonte: os autores.}
\label{fig:grafico_k2}
\end{figure}

É importante observar que o peso do incremento para $k$ é menor quando maior for o número de perguntas postadas seja por $A$ ou $B$. Dessa maneira, podemos depreender que a interação entre usuários novos, ou com poucas perguntas postadas, terá influência maior no crescimento do valor de $k$. Por outro lado, a interação entre usuários com muitas perguntas já postadas, incrementará um valor menor sobre $k$.

Tendo calculado um peso para cada aresta que liga os usuários, foi necessário definir um limiar $x$ que seria o valor mínimo de $k$ para considerar que dois usuários são similares o suficiente para serem postos em contato sob o mensageiro instantâneo.

A definição de $x$ provou-se um desafio, tendo em vista que o comportamento de $k$ varia em função do número total de perguntas postadas por dois usuários em análise. A figura \ref{fig:poucasquestoes} mostra que a convergência para $x = 1$ para o caso de de dois usuários que postaram poucas perguntas é rápida, pois o incremento de $k$ é maior. No caso simulado na figura \ref{fig:poucasquestões}, pouco mais de cinco respostas curtidas foram o suficiente para atingir o $x$.

\begin{figure}[!htb]
\centering
\includegraphics[width=14cm]{poucasquestoes.png}
\caption{Evolução de $k$ em função das respostas curtidas se $A$ e $B$ têm poucas perguntas postadas. Ex.: menos que 10 perguntas. Fonte: os autores.}
\label{fig:poucasquestoes}
\end{figure}

A figura \ref{fig:muitasquestoes}  demonstra que a convergência de $k$ e $x$ para usuários que já postaram muitas perguntas demanda muito mais respostas curtidas, neste caso mais que vinte respostas. Isto deve-se ao fato de ter sido decidido que a equação \ref{eq:k1} tem como divisor a soma de todas as perguntas já postadas pelos dois usuários, portanto, o incremento de $k$ é inversamente proporcional à soma de todas as perguntas postadas pelos usuários. 

\begin{figure}[!htb]
\centering
\includegraphics[width=14cm]{muitasquestoes.png}
\caption{Evolução de $k$ em função das respostas curtidas se $A$ e $B$ têm muitas perguntas postadas. Ex.: mais que 50 perguntas. Fonte: os autores.}
\label{fig:muitasquestoes}
\end{figure}

Uma adequação simplista para este problema seria definir $k$ como o número total de respostas curtidas entre dois usuários. Assim, $x$ seria o número mínimo de respostas que teriam que ser curtidas entre os usuários para que fossem considerados similares o suficiente para contatarem-se pelo mensageiro instantâneo. Porém, dessa maneira ignora-se que, se houverem muitas perguntas postadas por $A$ sem resposta de $B$ é possível que hajam mais diferenças do que similaridades entre os usuários, haja vista que muitas perguntas são ignoradas.

Portanto, é imprescindível considerar o volume de perguntas postadas. Uma evolução poderia adequar o software para contabilizar o número de perguntas postadas por $A$ e \emph{vistas} por $B$. Assim, poderíamos adequar o cálculo de $k$ como segue: 

\begin{equation}
k = k + \frac{R_{ab} + R_{ba}}{P_{AvB} + P_{BvA}} 
\label{eq:k2}
\end{equation}

Onde $R_{ab}$ é o número de respostas que o usuário $A$ recebeu do usuário $B$ e $A$ gostou; $R_{ba}$ é o número de respostas que o usuário $B$ recebeu do usuário $A$ e a $B$ gostou; $P_{AvB}$ é o número de perguntas postadas pelo usuário $A$ e vistas por $B$ e $P_{BvA}$ o número de respostas postadas pelo usuário $B$ e vistas por $A$.

Aplicando a equação \ref{eq:k2}, o incremento do valor de $k$ para cada resposta curtida seria menor para o caso de muitas perguntas terem sido vistas e poucas respostas terem sido dadas. De maneira inversa, quanto menor o conjunto de perguntas trocadas entre os usuários, maior o peso que uma resposta curtida terá sobre $k$.

Seria conveniente também, afetar as perguntas postadas com uma validade por cronologia ou fixar um valor máximo para o total de perguntas postadas pelos usuários $A$ e $B$. Para a implementação do produto neste trabalho, não foi considerada a idade das perguntas nem estabelecido um valor limite do somatório de perguntas postadas por $A$ e $B$ pois seria inconveniente para o desenvolvimento.

%=====================================================



		% metodologia
\chapter{Metodologia}
\label{cap:metodologia}

% figuras estão no subdiretório "figuras/" dentro deste capítulo
\graphicspath{\currfiledir/figuras/}

%=====================================================

\section{Método de desenvolvimento}

O produto foi desenvolvido pelo modelo de cascata. A priorização das atividades foi baseada na competência da equipe em cada área de desenvolvimento. Dessa maneira, o \emph{front end} foi considerado o foco e a prioridade no desenvolvimento, tendo em vista que o nível de conhecimento da equipe na área de design e usabilidade era perceptivelmente menor do que a habilidade para o desenvolvimento do \emph{back end}.

A análise de requisitos…

Um MVP, foi desenvolvido durante a fase de projeto. A figura \ref{fig:mvpCriarQuest} apresenta a tela de criação de perguntas do MVP. Neste produto, é possível criar perguntas (\ref{fig:mvpCriarQuest}), responder perguntas postadas por outros usuários (\ref{fig:mvpVerQuest}) e marcar essas respostas (\ref{fig:mvpVerResp}). Assim, o produto já tem informações suficientes para criar um grafo com os usuários no qual o peso das arestas é o nível de afinidade entre os eles baseado na quantidade de respostas marcadas.

\begin{figure}[!htb]
\centering
\includegraphics[width=14cm]{mvpCriarQuest.png}
\caption{Tela de criação de perguntas do MVP. Fonte: os autores.}
\label{fig:mvpCriarQuest}
\end{figure}

\begin{figure}[!htb]
\centering
\includegraphics[width=14cm]{mvpVerQuest.png}
\caption{Tela de visualização de perguntas do MVP. Nesta tela é possível escolher uma pergunta para ser respondida. Fonte: os autores.}
\label{fig:mvpVerQuest}
\end{figure}

\begin{figure}[!htb]
\centering
\includegraphics[width=14cm]{mvpVerResp.png}
\caption{Tela de visualização de respostas recebidas do MVP. Nesta tela é possível marcar perguntas favoritas. Fonte: os autores.}
\label{fig:mvpVerResp}
\end{figure}

\begin{figure}[!htb]
\centering
\includegraphics[width=14cm]{mvpVerGrafo.png}
\caption{Visualização do grafo que representa as ligações entre usuários da rede social no MVP. Fonte: os autores.}
\label{fig:mvpVerGrafo}
\end{figure}

\FloatBarrier


A figura \ref{fig:mvpVerGrafo} é a tela do MVP que representa os usuários e suas conexões criadas na rede social por meio de um grafo.


\section{Arquitetura}

\begin{itemize}
\item Qual a tecnologia
\item Custo
\item Arquitetura
\end{itemize}

Desenho da arquitetura

\begin{figure}[!htb]
\centering
\includegraphics[width=14cm]{arquitetura.png}
\caption{Arquitetura do software. Fonte: os autores.}
\label{fig:arquitetura}
\end{figure}

\FloatBarrier
%=====================================================

\section{Tecnologia aplicada}

\begin{itemize}
\item Linguagens de programação
\item Frameworks

\end{itemize}

Frameworks: IONIC django

angular e github.

Linguagens: python typescript javascript html/css
Hardware: PC Notebook thinkpad x201, SSD 120GB e 8GB de RAM primeira geracao do i5 e arch linux, server usa debian stretch
Infraestrutura: 	Docker
VPS: CPU: Intel (Haswell, noTSX) (1)@2.3GHz GPU: CirrusLogicGD5446 Memory: 583MB / 1956MB.



%=====================================================

\section{Análise do sistema}

Diagramas de caso de USO
\begin{figure}[!htb]
\centering
\includegraphics[width=16cm]{DCU1.png}
\caption{Diagrama de caso de uso nível 1. Fonte: os autores.}
\label{fig:DCU1}
\end{figure}

\begin{figure}[!htb]
\centering
\includegraphics[width=16cm]{DCU2.png}
\caption{Diagrama de caso de uso nível 2. Fonte: os autores.}
\label{fig:DCU2}
\end{figure}
%=====================================================

Diagramas de sequência

\begin{figure}[!htb]
\centering
\includegraphics[width=16cm]{UC001-VisualizarQuestao.png}
\caption{Diagrama de caso de uso UC001 - Visualizar Questão. Fonte: os autores.}
\label{fig:UC001}
\end{figure}

\begin{figure}[!htb]
\centering
\includegraphics[width=16cm]{UC002-ResponderQuestao.png}
\caption{Diagrama de caso de uso UC002 - Responder Questão. Fonte: os autores.}
\label{fig:UC002}
\end{figure}

\begin{figure}[!htb]
\centering
\includegraphics[width=16cm]{UC003-VisualizarRespostas.png}
\caption{Diagrama de caso de uso UC003 - Visualizar Respostas. Fonte: os autores.}
\label{fig:UC003}
\end{figure}


\begin{figure}[!htb]
\centering
\includegraphics[width=16cm]{UC004-CriarQuestao.png}
\caption{Diagrama de caso de uso UC004 - Criar Questão. Fonte: os autores.}
\label{fig:UC004}
\end{figure}


\begin{figure}[!htb]
\centering
\includegraphics[width=16cm]{UC005-VisualizarCombinacoes.png}
\caption{Diagrama de caso de uso UC005 - Visualizar Combinações. Fonte: os autores.}
\label{fig:UC005}
\end{figure}


Diagrama de Classes

\begin{figure}[!htb]
\centering
\includegraphics[width=16cm]{DiagramaClasse.png}
\caption{Diagrama de classes. Fonte: os autores.}
\label{fig:diagramaClasse}
\end{figure}
\FloatBarrier



%=====================================================

\section{Cálculo de do peso da aresta}
O peso de uma aresta do grafo, $k$, representa o nível de afinidade entre os dois usuários conectados por esta aresta. Um usuário pode estar conectado a vários outros usuários.

Quando um usuário entra pela primeira vez na rede social, ele é obrigado a preencher um questionário contendo as questões constantes na tabela \ref{tab:questoes}. A partir das respostas deste questionário, um valor para $k$ é calculado levando em conta, tão somente, a similaridade entre as respostas de cada usuário.

\begin{table}[!htp]
\centering
\caption{Formulário inicial}
\label{tab:questoes}
\begin{tabular}{ || c ||}
\hline
Pergunta\\
\hline
\hline
Você prefere cachorro ou gato?\\  
\hline
Você prefere rock ou funk?\\
\hline
Você prefere verão ou inverno? \\
\hline
Você prefere cinema ou teatro?\\
\hline
Você prefere cerveja ou vinho?\\
\hline
Você prefere o dia ou a noite?\\
\hline
Você prefere sair ou ficar em casa?\\
\hline
Você fuma?\\
\hline
Você tem alguma religião?\\
\hline
Você acredita em signos?\\
\hline
Você prefere praia ou campo?\\
\hline
\end{tabular}
\end{table}

Para a determinação inicial de $k$, logo após o preenchimento do formulário, é calculado o total de respostas iguais entre dois usuários e aplicada a seguinte equação:

\begin{equation}
k_{ab} = \frac{R_{ab}*(x-1)}{N_{p}}
 \label{eq:k0}
\end{equation}


Onde $R_{ab}$ é o total de respostas do usuário A iguais ao usuário B; $x$ é o valor objetivo para considerar dois usuários similares - este valor será discutido adiante nesta seção - e $N_{p}$ é o total de perguntas do questionário inicial.

Na equação \ref{eq:k0}, o numerador é multiplicado por $x-1$ para que os usuários que tiveram todas as questões respondidas da mesma maneira no questionário tão somente fiquem muito próximos da margem que define a habilitação do mensageiro instantâneo. O objetivo é tornar obrigatória a interação por meio de perguntas e respostas antes que dois usuários possam ser considerados similares o suficiente para desfrutarem do mensageiro.

Então, conforme as perguntas postadas pelos usuários vão sendo respondidas e apreciadas, valor do peso da aresta, definido como $k_{ab}$, que liga os usuários $A$ e $B$, é recalculado, para cada resposta apontada como apreciada, a partir da seguinte definição:

\begin{equation}
k_{ab} = k_{ab} + \frac{(R_{ab} + R_{ba})}{(P_{A} + P_{B})}
\label{eq:k1}
\end{equation}

Onde $R_{ab}$ é o número de respostas que o usuário $A$ recebeu do usuário $B$ e $A$ gostou; $R_{ba}$ é o número de respostas que o usuário $B$ recebeu do usuário $A$ e a $B$ gostou; $P_{A}$ é o número de perguntas postadas pelo usuário $A$ e $P_{B}$ o número de respostas postadas pelo usuário $B$.

Na figura \ref{fig:grafico_k1}, podemos ver a evolução do peso $k$ para cada resposta recebida pelo usuário $A$ que ele gostou. Nesta situação o crescimento do valor de $k$ é linear, pois todas as perguntas postadas pelo usuário $A$ estão respondidas por $B$ e as respectivas respostas foram curtidas.

\begin{figure}[!htb]
\centering
\includegraphics[width=14cm]{grafico_k1.png}
\caption{Evolução de $k$ se todas as perguntas postadas por $A$ forem respondidas por $B$. Fonte: os autores.}
\label{fig:grafico_k1}
\end{figure}

Como o cálculo leva em conta todas as perguntas postadas pelo usuário $A$, no caso de já haver perguntas postadas e não respondidas, o crescimento do valor de $k$ é afetado num grau inversamente proporcional ao número de perguntas postadas, como pode-se observar na figura \ref{fig:grafico_k2}.

\begin{figure}[!htb]
\centering
\includegraphics[width=14cm]{grafico_k2.png}
\caption{Evolução de $k$ se $A$ postou 50 perguntas e apenas 25 respostas de $B$ foram curtidas. Fonte: os autores.}
\label{fig:grafico_k2}
\end{figure}

É importante observar que o peso do incremento para $k$ é menor quando maior for o número de perguntas postadas seja por $A$ ou $B$. Dessa maneira, podemos depreender que a interação entre usuários novos, ou com poucas perguntas postadas, terá influência maior no crescimento do valor de $k$. Por outro lado, a interação entre usuários com muitas perguntas já postadas, incrementará um valor menor sobre $k$.

Tendo calculado um peso para cada aresta que liga os usuários, foi necessário definir um limiar $x$ que seria o valor mínimo de $k$ para considerar que dois usuários são similares o suficiente para serem postos em contato sob o mensageiro instantâneo.

A definição de $x$ provou-se um desafio, tendo em vista que o comportamento de $k$ varia em função do número total de perguntas postadas por dois usuários em análise. A figura \ref{fig:poucasquestoes} mostra que a convergência para $x = 1$ para o caso de de dois usuários que postaram poucas perguntas é rápida, pois o incremento de $k$ é maior. No caso simulado na figura \ref{fig:poucasquestões}, pouco mais de cinco respostas curtidas foram o suficiente para atingir o $x$.

\begin{figure}[!htb]
\centering
\includegraphics[width=14cm]{poucasquestoes.png}
\caption{Evolução de $k$ em função das respostas curtidas se $A$ e $B$ têm poucas perguntas postadas. Ex.: menos que 10 perguntas. Fonte: os autores.}
\label{fig:poucasquestoes}
\end{figure}

A figura \ref{fig:muitasquestoes}  demonstra que a convergência de $k$ e $x$ para usuários que já postaram muitas perguntas demanda muito mais respostas curtidas, neste caso mais que vinte respostas. Isto deve-se ao fato de ter sido decidido que a equação \ref{eq:k1} tem como divisor a soma de todas as perguntas já postadas pelos dois usuários, portanto, o incremento de $k$ é inversamente proporcional à soma de todas as perguntas postadas pelos usuários. 

\begin{figure}[!htb]
\centering
\includegraphics[width=14cm]{muitasquestoes.png}
\caption{Evolução de $k$ em função das respostas curtidas se $A$ e $B$ têm muitas perguntas postadas. Ex.: mais que 50 perguntas. Fonte: os autores.}
\label{fig:muitasquestoes}
\end{figure}

Uma adequação simplista para este problema seria definir $k$ como o número total de respostas curtidas entre dois usuários. Assim, $x$ seria o número mínimo de respostas que teriam que ser curtidas entre os usuários para que fossem considerados similares o suficiente para contatarem-se pelo mensageiro instantâneo. Porém, dessa maneira ignora-se que, se houverem muitas perguntas postadas por $A$ sem resposta de $B$ é possível que hajam mais diferenças do que similaridades entre os usuários, haja vista que muitas perguntas são ignoradas.

Portanto, é imprescindível considerar o volume de perguntas postadas. Uma evolução poderia adequar o software para contabilizar o número de perguntas postadas por $A$ e \emph{vistas} por $B$. Assim, poderíamos adequar o cálculo de $k$ como segue: 

\begin{equation}
k = k + \frac{R_{ab} + R_{ba}}{P_{AvB} + P_{BvA}} 
\label{eq:k2}
\end{equation}

Onde $R_{ab}$ é o número de respostas que o usuário $A$ recebeu do usuário $B$ e $A$ gostou; $R_{ba}$ é o número de respostas que o usuário $B$ recebeu do usuário $A$ e a $B$ gostou; $P_{AvB}$ é o número de perguntas postadas pelo usuário $A$ e vistas por $B$ e $P_{BvA}$ o número de respostas postadas pelo usuário $B$ e vistas por $A$.

Aplicando a equação \ref{eq:k2}, o incremento do valor de $k$ para cada resposta curtida seria menor para o caso de muitas perguntas terem sido vistas e poucas respostas terem sido dadas. De maneira inversa, quanto menor o conjunto de perguntas trocadas entre os usuários, maior o peso que uma resposta curtida terá sobre $k$.

Seria conveniente também, afetar as perguntas postadas com uma validade por cronologia ou fixar um valor máximo para o total de perguntas postadas pelos usuários $A$ e $B$. Para a implementação do produto neste trabalho, não foi considerada a idade das perguntas nem estabelecido um valor limite do somatório de perguntas postadas por $A$ e $B$ pois seria inconveniente para o desenvolvimento.

%=====================================================



		% produto final
%\chapter{Metodologia}
\label{cap:metodologia}

% figuras estão no subdiretório "figuras/" dentro deste capítulo
\graphicspath{\currfiledir/figuras/}

%=====================================================

\section{Método de desenvolvimento}

O produto foi desenvolvido pelo modelo de cascata. A priorização das atividades foi baseada na competência da equipe em cada área de desenvolvimento. Dessa maneira, o \emph{front end} foi considerado o foco e a prioridade no desenvolvimento, tendo em vista que o nível de conhecimento da equipe na área de design e usabilidade era perceptivelmente menor do que a habilidade para o desenvolvimento do \emph{back end}.

A análise de requisitos…

Um MVP, foi desenvolvido durante a fase de projeto. A figura \ref{fig:mvpCriarQuest} apresenta a tela de criação de perguntas do MVP. Neste produto, é possível criar perguntas (\ref{fig:mvpCriarQuest}), responder perguntas postadas por outros usuários (\ref{fig:mvpVerQuest}) e marcar essas respostas (\ref{fig:mvpVerResp}). Assim, o produto já tem informações suficientes para criar um grafo com os usuários no qual o peso das arestas é o nível de afinidade entre os eles baseado na quantidade de respostas marcadas.

\begin{figure}[!htb]
\centering
\includegraphics[width=14cm]{mvpCriarQuest.png}
\caption{Tela de criação de perguntas do MVP. Fonte: os autores.}
\label{fig:mvpCriarQuest}
\end{figure}

\begin{figure}[!htb]
\centering
\includegraphics[width=14cm]{mvpVerQuest.png}
\caption{Tela de visualização de perguntas do MVP. Nesta tela é possível escolher uma pergunta para ser respondida. Fonte: os autores.}
\label{fig:mvpVerQuest}
\end{figure}

\begin{figure}[!htb]
\centering
\includegraphics[width=14cm]{mvpVerResp.png}
\caption{Tela de visualização de respostas recebidas do MVP. Nesta tela é possível marcar perguntas favoritas. Fonte: os autores.}
\label{fig:mvpVerResp}
\end{figure}

\begin{figure}[!htb]
\centering
\includegraphics[width=14cm]{mvpVerGrafo.png}
\caption{Visualização do grafo que representa as ligações entre usuários da rede social no MVP. Fonte: os autores.}
\label{fig:mvpVerGrafo}
\end{figure}

\FloatBarrier


A figura \ref{fig:mvpVerGrafo} é a tela do MVP que representa os usuários e suas conexões criadas na rede social por meio de um grafo.


\section{Arquitetura}

\begin{itemize}
\item Qual a tecnologia
\item Custo
\item Arquitetura
\end{itemize}

Desenho da arquitetura

\begin{figure}[!htb]
\centering
\includegraphics[width=14cm]{arquitetura.png}
\caption{Arquitetura do software. Fonte: os autores.}
\label{fig:arquitetura}
\end{figure}

\FloatBarrier
%=====================================================

\section{Tecnologia aplicada}

\begin{itemize}
\item Linguagens de programação
\item Frameworks

\end{itemize}

Frameworks: IONIC django

angular e github.

Linguagens: python typescript javascript html/css
Hardware: PC Notebook thinkpad x201, SSD 120GB e 8GB de RAM primeira geracao do i5 e arch linux, server usa debian stretch
Infraestrutura: 	Docker
VPS: CPU: Intel (Haswell, noTSX) (1)@2.3GHz GPU: CirrusLogicGD5446 Memory: 583MB / 1956MB.



%=====================================================

\section{Análise do sistema}

Diagramas de caso de USO
\begin{figure}[!htb]
\centering
\includegraphics[width=16cm]{DCU1.png}
\caption{Diagrama de caso de uso nível 1. Fonte: os autores.}
\label{fig:DCU1}
\end{figure}

\begin{figure}[!htb]
\centering
\includegraphics[width=16cm]{DCU2.png}
\caption{Diagrama de caso de uso nível 2. Fonte: os autores.}
\label{fig:DCU2}
\end{figure}
%=====================================================

Diagramas de sequência

\begin{figure}[!htb]
\centering
\includegraphics[width=16cm]{UC001-VisualizarQuestao.png}
\caption{Diagrama de caso de uso UC001 - Visualizar Questão. Fonte: os autores.}
\label{fig:UC001}
\end{figure}

\begin{figure}[!htb]
\centering
\includegraphics[width=16cm]{UC002-ResponderQuestao.png}
\caption{Diagrama de caso de uso UC002 - Responder Questão. Fonte: os autores.}
\label{fig:UC002}
\end{figure}

\begin{figure}[!htb]
\centering
\includegraphics[width=16cm]{UC003-VisualizarRespostas.png}
\caption{Diagrama de caso de uso UC003 - Visualizar Respostas. Fonte: os autores.}
\label{fig:UC003}
\end{figure}


\begin{figure}[!htb]
\centering
\includegraphics[width=16cm]{UC004-CriarQuestao.png}
\caption{Diagrama de caso de uso UC004 - Criar Questão. Fonte: os autores.}
\label{fig:UC004}
\end{figure}


\begin{figure}[!htb]
\centering
\includegraphics[width=16cm]{UC005-VisualizarCombinacoes.png}
\caption{Diagrama de caso de uso UC005 - Visualizar Combinações. Fonte: os autores.}
\label{fig:UC005}
\end{figure}


Diagrama de Classes

\begin{figure}[!htb]
\centering
\includegraphics[width=16cm]{DiagramaClasse.png}
\caption{Diagrama de classes. Fonte: os autores.}
\label{fig:diagramaClasse}
\end{figure}
\FloatBarrier



%=====================================================

\section{Cálculo de do peso da aresta}
O peso de uma aresta do grafo, $k$, representa o nível de afinidade entre os dois usuários conectados por esta aresta. Um usuário pode estar conectado a vários outros usuários.

Quando um usuário entra pela primeira vez na rede social, ele é obrigado a preencher um questionário contendo as questões constantes na tabela \ref{tab:questoes}. A partir das respostas deste questionário, um valor para $k$ é calculado levando em conta, tão somente, a similaridade entre as respostas de cada usuário.

\begin{table}[!htp]
\centering
\caption{Formulário inicial}
\label{tab:questoes}
\begin{tabular}{ || c ||}
\hline
Pergunta\\
\hline
\hline
Você prefere cachorro ou gato?\\  
\hline
Você prefere rock ou funk?\\
\hline
Você prefere verão ou inverno? \\
\hline
Você prefere cinema ou teatro?\\
\hline
Você prefere cerveja ou vinho?\\
\hline
Você prefere o dia ou a noite?\\
\hline
Você prefere sair ou ficar em casa?\\
\hline
Você fuma?\\
\hline
Você tem alguma religião?\\
\hline
Você acredita em signos?\\
\hline
Você prefere praia ou campo?\\
\hline
\end{tabular}
\end{table}

Para a determinação inicial de $k$, logo após o preenchimento do formulário, é calculado o total de respostas iguais entre dois usuários e aplicada a seguinte equação:

\begin{equation}
k_{ab} = \frac{R_{ab}*(x-1)}{N_{p}}
 \label{eq:k0}
\end{equation}


Onde $R_{ab}$ é o total de respostas do usuário A iguais ao usuário B; $x$ é o valor objetivo para considerar dois usuários similares - este valor será discutido adiante nesta seção - e $N_{p}$ é o total de perguntas do questionário inicial.

Na equação \ref{eq:k0}, o numerador é multiplicado por $x-1$ para que os usuários que tiveram todas as questões respondidas da mesma maneira no questionário tão somente fiquem muito próximos da margem que define a habilitação do mensageiro instantâneo. O objetivo é tornar obrigatória a interação por meio de perguntas e respostas antes que dois usuários possam ser considerados similares o suficiente para desfrutarem do mensageiro.

Então, conforme as perguntas postadas pelos usuários vão sendo respondidas e apreciadas, valor do peso da aresta, definido como $k_{ab}$, que liga os usuários $A$ e $B$, é recalculado, para cada resposta apontada como apreciada, a partir da seguinte definição:

\begin{equation}
k_{ab} = k_{ab} + \frac{(R_{ab} + R_{ba})}{(P_{A} + P_{B})}
\label{eq:k1}
\end{equation}

Onde $R_{ab}$ é o número de respostas que o usuário $A$ recebeu do usuário $B$ e $A$ gostou; $R_{ba}$ é o número de respostas que o usuário $B$ recebeu do usuário $A$ e a $B$ gostou; $P_{A}$ é o número de perguntas postadas pelo usuário $A$ e $P_{B}$ o número de respostas postadas pelo usuário $B$.

Na figura \ref{fig:grafico_k1}, podemos ver a evolução do peso $k$ para cada resposta recebida pelo usuário $A$ que ele gostou. Nesta situação o crescimento do valor de $k$ é linear, pois todas as perguntas postadas pelo usuário $A$ estão respondidas por $B$ e as respectivas respostas foram curtidas.

\begin{figure}[!htb]
\centering
\includegraphics[width=14cm]{grafico_k1.png}
\caption{Evolução de $k$ se todas as perguntas postadas por $A$ forem respondidas por $B$. Fonte: os autores.}
\label{fig:grafico_k1}
\end{figure}

Como o cálculo leva em conta todas as perguntas postadas pelo usuário $A$, no caso de já haver perguntas postadas e não respondidas, o crescimento do valor de $k$ é afetado num grau inversamente proporcional ao número de perguntas postadas, como pode-se observar na figura \ref{fig:grafico_k2}.

\begin{figure}[!htb]
\centering
\includegraphics[width=14cm]{grafico_k2.png}
\caption{Evolução de $k$ se $A$ postou 50 perguntas e apenas 25 respostas de $B$ foram curtidas. Fonte: os autores.}
\label{fig:grafico_k2}
\end{figure}

É importante observar que o peso do incremento para $k$ é menor quando maior for o número de perguntas postadas seja por $A$ ou $B$. Dessa maneira, podemos depreender que a interação entre usuários novos, ou com poucas perguntas postadas, terá influência maior no crescimento do valor de $k$. Por outro lado, a interação entre usuários com muitas perguntas já postadas, incrementará um valor menor sobre $k$.

Tendo calculado um peso para cada aresta que liga os usuários, foi necessário definir um limiar $x$ que seria o valor mínimo de $k$ para considerar que dois usuários são similares o suficiente para serem postos em contato sob o mensageiro instantâneo.

A definição de $x$ provou-se um desafio, tendo em vista que o comportamento de $k$ varia em função do número total de perguntas postadas por dois usuários em análise. A figura \ref{fig:poucasquestoes} mostra que a convergência para $x = 1$ para o caso de de dois usuários que postaram poucas perguntas é rápida, pois o incremento de $k$ é maior. No caso simulado na figura \ref{fig:poucasquestões}, pouco mais de cinco respostas curtidas foram o suficiente para atingir o $x$.

\begin{figure}[!htb]
\centering
\includegraphics[width=14cm]{poucasquestoes.png}
\caption{Evolução de $k$ em função das respostas curtidas se $A$ e $B$ têm poucas perguntas postadas. Ex.: menos que 10 perguntas. Fonte: os autores.}
\label{fig:poucasquestoes}
\end{figure}

A figura \ref{fig:muitasquestoes}  demonstra que a convergência de $k$ e $x$ para usuários que já postaram muitas perguntas demanda muito mais respostas curtidas, neste caso mais que vinte respostas. Isto deve-se ao fato de ter sido decidido que a equação \ref{eq:k1} tem como divisor a soma de todas as perguntas já postadas pelos dois usuários, portanto, o incremento de $k$ é inversamente proporcional à soma de todas as perguntas postadas pelos usuários. 

\begin{figure}[!htb]
\centering
\includegraphics[width=14cm]{muitasquestoes.png}
\caption{Evolução de $k$ em função das respostas curtidas se $A$ e $B$ têm muitas perguntas postadas. Ex.: mais que 50 perguntas. Fonte: os autores.}
\label{fig:muitasquestoes}
\end{figure}

Uma adequação simplista para este problema seria definir $k$ como o número total de respostas curtidas entre dois usuários. Assim, $x$ seria o número mínimo de respostas que teriam que ser curtidas entre os usuários para que fossem considerados similares o suficiente para contatarem-se pelo mensageiro instantâneo. Porém, dessa maneira ignora-se que, se houverem muitas perguntas postadas por $A$ sem resposta de $B$ é possível que hajam mais diferenças do que similaridades entre os usuários, haja vista que muitas perguntas são ignoradas.

Portanto, é imprescindível considerar o volume de perguntas postadas. Uma evolução poderia adequar o software para contabilizar o número de perguntas postadas por $A$ e \emph{vistas} por $B$. Assim, poderíamos adequar o cálculo de $k$ como segue: 

\begin{equation}
k = k + \frac{R_{ab} + R_{ba}}{P_{AvB} + P_{BvA}} 
\label{eq:k2}
\end{equation}

Onde $R_{ab}$ é o número de respostas que o usuário $A$ recebeu do usuário $B$ e $A$ gostou; $R_{ba}$ é o número de respostas que o usuário $B$ recebeu do usuário $A$ e a $B$ gostou; $P_{AvB}$ é o número de perguntas postadas pelo usuário $A$ e vistas por $B$ e $P_{BvA}$ o número de respostas postadas pelo usuário $B$ e vistas por $A$.

Aplicando a equação \ref{eq:k2}, o incremento do valor de $k$ para cada resposta curtida seria menor para o caso de muitas perguntas terem sido vistas e poucas respostas terem sido dadas. De maneira inversa, quanto menor o conjunto de perguntas trocadas entre os usuários, maior o peso que uma resposta curtida terá sobre $k$.

Seria conveniente também, afetar as perguntas postadas com uma validade por cronologia ou fixar um valor máximo para o total de perguntas postadas pelos usuários $A$ e $B$. Para a implementação do produto neste trabalho, não foi considerada a idade das perguntas nem estabelecido um valor limite do somatório de perguntas postadas por $A$ e $B$ pois seria inconveniente para o desenvolvimento.

%=====================================================



		% experimentação e validação
\chapter{Metodologia}
\label{cap:metodologia}

% figuras estão no subdiretório "figuras/" dentro deste capítulo
\graphicspath{\currfiledir/figuras/}

%=====================================================

\section{Método de desenvolvimento}

O produto foi desenvolvido pelo modelo de cascata. A priorização das atividades foi baseada na competência da equipe em cada área de desenvolvimento. Dessa maneira, o \emph{front end} foi considerado o foco e a prioridade no desenvolvimento, tendo em vista que o nível de conhecimento da equipe na área de design e usabilidade era perceptivelmente menor do que a habilidade para o desenvolvimento do \emph{back end}.

A análise de requisitos…

Um MVP, foi desenvolvido durante a fase de projeto. A figura \ref{fig:mvpCriarQuest} apresenta a tela de criação de perguntas do MVP. Neste produto, é possível criar perguntas (\ref{fig:mvpCriarQuest}), responder perguntas postadas por outros usuários (\ref{fig:mvpVerQuest}) e marcar essas respostas (\ref{fig:mvpVerResp}). Assim, o produto já tem informações suficientes para criar um grafo com os usuários no qual o peso das arestas é o nível de afinidade entre os eles baseado na quantidade de respostas marcadas.

\begin{figure}[!htb]
\centering
\includegraphics[width=14cm]{mvpCriarQuest.png}
\caption{Tela de criação de perguntas do MVP. Fonte: os autores.}
\label{fig:mvpCriarQuest}
\end{figure}

\begin{figure}[!htb]
\centering
\includegraphics[width=14cm]{mvpVerQuest.png}
\caption{Tela de visualização de perguntas do MVP. Nesta tela é possível escolher uma pergunta para ser respondida. Fonte: os autores.}
\label{fig:mvpVerQuest}
\end{figure}

\begin{figure}[!htb]
\centering
\includegraphics[width=14cm]{mvpVerResp.png}
\caption{Tela de visualização de respostas recebidas do MVP. Nesta tela é possível marcar perguntas favoritas. Fonte: os autores.}
\label{fig:mvpVerResp}
\end{figure}

\begin{figure}[!htb]
\centering
\includegraphics[width=14cm]{mvpVerGrafo.png}
\caption{Visualização do grafo que representa as ligações entre usuários da rede social no MVP. Fonte: os autores.}
\label{fig:mvpVerGrafo}
\end{figure}

\FloatBarrier


A figura \ref{fig:mvpVerGrafo} é a tela do MVP que representa os usuários e suas conexões criadas na rede social por meio de um grafo.


\section{Arquitetura}

\begin{itemize}
\item Qual a tecnologia
\item Custo
\item Arquitetura
\end{itemize}

Desenho da arquitetura

\begin{figure}[!htb]
\centering
\includegraphics[width=14cm]{arquitetura.png}
\caption{Arquitetura do software. Fonte: os autores.}
\label{fig:arquitetura}
\end{figure}

\FloatBarrier
%=====================================================

\section{Tecnologia aplicada}

\begin{itemize}
\item Linguagens de programação
\item Frameworks

\end{itemize}

Frameworks: IONIC django

angular e github.

Linguagens: python typescript javascript html/css
Hardware: PC Notebook thinkpad x201, SSD 120GB e 8GB de RAM primeira geracao do i5 e arch linux, server usa debian stretch
Infraestrutura: 	Docker
VPS: CPU: Intel (Haswell, noTSX) (1)@2.3GHz GPU: CirrusLogicGD5446 Memory: 583MB / 1956MB.



%=====================================================

\section{Análise do sistema}

Diagramas de caso de USO
\begin{figure}[!htb]
\centering
\includegraphics[width=16cm]{DCU1.png}
\caption{Diagrama de caso de uso nível 1. Fonte: os autores.}
\label{fig:DCU1}
\end{figure}

\begin{figure}[!htb]
\centering
\includegraphics[width=16cm]{DCU2.png}
\caption{Diagrama de caso de uso nível 2. Fonte: os autores.}
\label{fig:DCU2}
\end{figure}
%=====================================================

Diagramas de sequência

\begin{figure}[!htb]
\centering
\includegraphics[width=16cm]{UC001-VisualizarQuestao.png}
\caption{Diagrama de caso de uso UC001 - Visualizar Questão. Fonte: os autores.}
\label{fig:UC001}
\end{figure}

\begin{figure}[!htb]
\centering
\includegraphics[width=16cm]{UC002-ResponderQuestao.png}
\caption{Diagrama de caso de uso UC002 - Responder Questão. Fonte: os autores.}
\label{fig:UC002}
\end{figure}

\begin{figure}[!htb]
\centering
\includegraphics[width=16cm]{UC003-VisualizarRespostas.png}
\caption{Diagrama de caso de uso UC003 - Visualizar Respostas. Fonte: os autores.}
\label{fig:UC003}
\end{figure}


\begin{figure}[!htb]
\centering
\includegraphics[width=16cm]{UC004-CriarQuestao.png}
\caption{Diagrama de caso de uso UC004 - Criar Questão. Fonte: os autores.}
\label{fig:UC004}
\end{figure}


\begin{figure}[!htb]
\centering
\includegraphics[width=16cm]{UC005-VisualizarCombinacoes.png}
\caption{Diagrama de caso de uso UC005 - Visualizar Combinações. Fonte: os autores.}
\label{fig:UC005}
\end{figure}


Diagrama de Classes

\begin{figure}[!htb]
\centering
\includegraphics[width=16cm]{DiagramaClasse.png}
\caption{Diagrama de classes. Fonte: os autores.}
\label{fig:diagramaClasse}
\end{figure}
\FloatBarrier



%=====================================================

\section{Cálculo de do peso da aresta}
O peso de uma aresta do grafo, $k$, representa o nível de afinidade entre os dois usuários conectados por esta aresta. Um usuário pode estar conectado a vários outros usuários.

Quando um usuário entra pela primeira vez na rede social, ele é obrigado a preencher um questionário contendo as questões constantes na tabela \ref{tab:questoes}. A partir das respostas deste questionário, um valor para $k$ é calculado levando em conta, tão somente, a similaridade entre as respostas de cada usuário.

\begin{table}[!htp]
\centering
\caption{Formulário inicial}
\label{tab:questoes}
\begin{tabular}{ || c ||}
\hline
Pergunta\\
\hline
\hline
Você prefere cachorro ou gato?\\  
\hline
Você prefere rock ou funk?\\
\hline
Você prefere verão ou inverno? \\
\hline
Você prefere cinema ou teatro?\\
\hline
Você prefere cerveja ou vinho?\\
\hline
Você prefere o dia ou a noite?\\
\hline
Você prefere sair ou ficar em casa?\\
\hline
Você fuma?\\
\hline
Você tem alguma religião?\\
\hline
Você acredita em signos?\\
\hline
Você prefere praia ou campo?\\
\hline
\end{tabular}
\end{table}

Para a determinação inicial de $k$, logo após o preenchimento do formulário, é calculado o total de respostas iguais entre dois usuários e aplicada a seguinte equação:

\begin{equation}
k_{ab} = \frac{R_{ab}*(x-1)}{N_{p}}
 \label{eq:k0}
\end{equation}


Onde $R_{ab}$ é o total de respostas do usuário A iguais ao usuário B; $x$ é o valor objetivo para considerar dois usuários similares - este valor será discutido adiante nesta seção - e $N_{p}$ é o total de perguntas do questionário inicial.

Na equação \ref{eq:k0}, o numerador é multiplicado por $x-1$ para que os usuários que tiveram todas as questões respondidas da mesma maneira no questionário tão somente fiquem muito próximos da margem que define a habilitação do mensageiro instantâneo. O objetivo é tornar obrigatória a interação por meio de perguntas e respostas antes que dois usuários possam ser considerados similares o suficiente para desfrutarem do mensageiro.

Então, conforme as perguntas postadas pelos usuários vão sendo respondidas e apreciadas, valor do peso da aresta, definido como $k_{ab}$, que liga os usuários $A$ e $B$, é recalculado, para cada resposta apontada como apreciada, a partir da seguinte definição:

\begin{equation}
k_{ab} = k_{ab} + \frac{(R_{ab} + R_{ba})}{(P_{A} + P_{B})}
\label{eq:k1}
\end{equation}

Onde $R_{ab}$ é o número de respostas que o usuário $A$ recebeu do usuário $B$ e $A$ gostou; $R_{ba}$ é o número de respostas que o usuário $B$ recebeu do usuário $A$ e a $B$ gostou; $P_{A}$ é o número de perguntas postadas pelo usuário $A$ e $P_{B}$ o número de respostas postadas pelo usuário $B$.

Na figura \ref{fig:grafico_k1}, podemos ver a evolução do peso $k$ para cada resposta recebida pelo usuário $A$ que ele gostou. Nesta situação o crescimento do valor de $k$ é linear, pois todas as perguntas postadas pelo usuário $A$ estão respondidas por $B$ e as respectivas respostas foram curtidas.

\begin{figure}[!htb]
\centering
\includegraphics[width=14cm]{grafico_k1.png}
\caption{Evolução de $k$ se todas as perguntas postadas por $A$ forem respondidas por $B$. Fonte: os autores.}
\label{fig:grafico_k1}
\end{figure}

Como o cálculo leva em conta todas as perguntas postadas pelo usuário $A$, no caso de já haver perguntas postadas e não respondidas, o crescimento do valor de $k$ é afetado num grau inversamente proporcional ao número de perguntas postadas, como pode-se observar na figura \ref{fig:grafico_k2}.

\begin{figure}[!htb]
\centering
\includegraphics[width=14cm]{grafico_k2.png}
\caption{Evolução de $k$ se $A$ postou 50 perguntas e apenas 25 respostas de $B$ foram curtidas. Fonte: os autores.}
\label{fig:grafico_k2}
\end{figure}

É importante observar que o peso do incremento para $k$ é menor quando maior for o número de perguntas postadas seja por $A$ ou $B$. Dessa maneira, podemos depreender que a interação entre usuários novos, ou com poucas perguntas postadas, terá influência maior no crescimento do valor de $k$. Por outro lado, a interação entre usuários com muitas perguntas já postadas, incrementará um valor menor sobre $k$.

Tendo calculado um peso para cada aresta que liga os usuários, foi necessário definir um limiar $x$ que seria o valor mínimo de $k$ para considerar que dois usuários são similares o suficiente para serem postos em contato sob o mensageiro instantâneo.

A definição de $x$ provou-se um desafio, tendo em vista que o comportamento de $k$ varia em função do número total de perguntas postadas por dois usuários em análise. A figura \ref{fig:poucasquestoes} mostra que a convergência para $x = 1$ para o caso de de dois usuários que postaram poucas perguntas é rápida, pois o incremento de $k$ é maior. No caso simulado na figura \ref{fig:poucasquestões}, pouco mais de cinco respostas curtidas foram o suficiente para atingir o $x$.

\begin{figure}[!htb]
\centering
\includegraphics[width=14cm]{poucasquestoes.png}
\caption{Evolução de $k$ em função das respostas curtidas se $A$ e $B$ têm poucas perguntas postadas. Ex.: menos que 10 perguntas. Fonte: os autores.}
\label{fig:poucasquestoes}
\end{figure}

A figura \ref{fig:muitasquestoes}  demonstra que a convergência de $k$ e $x$ para usuários que já postaram muitas perguntas demanda muito mais respostas curtidas, neste caso mais que vinte respostas. Isto deve-se ao fato de ter sido decidido que a equação \ref{eq:k1} tem como divisor a soma de todas as perguntas já postadas pelos dois usuários, portanto, o incremento de $k$ é inversamente proporcional à soma de todas as perguntas postadas pelos usuários. 

\begin{figure}[!htb]
\centering
\includegraphics[width=14cm]{muitasquestoes.png}
\caption{Evolução de $k$ em função das respostas curtidas se $A$ e $B$ têm muitas perguntas postadas. Ex.: mais que 50 perguntas. Fonte: os autores.}
\label{fig:muitasquestoes}
\end{figure}

Uma adequação simplista para este problema seria definir $k$ como o número total de respostas curtidas entre dois usuários. Assim, $x$ seria o número mínimo de respostas que teriam que ser curtidas entre os usuários para que fossem considerados similares o suficiente para contatarem-se pelo mensageiro instantâneo. Porém, dessa maneira ignora-se que, se houverem muitas perguntas postadas por $A$ sem resposta de $B$ é possível que hajam mais diferenças do que similaridades entre os usuários, haja vista que muitas perguntas são ignoradas.

Portanto, é imprescindível considerar o volume de perguntas postadas. Uma evolução poderia adequar o software para contabilizar o número de perguntas postadas por $A$ e \emph{vistas} por $B$. Assim, poderíamos adequar o cálculo de $k$ como segue: 

\begin{equation}
k = k + \frac{R_{ab} + R_{ba}}{P_{AvB} + P_{BvA}} 
\label{eq:k2}
\end{equation}

Onde $R_{ab}$ é o número de respostas que o usuário $A$ recebeu do usuário $B$ e $A$ gostou; $R_{ba}$ é o número de respostas que o usuário $B$ recebeu do usuário $A$ e a $B$ gostou; $P_{AvB}$ é o número de perguntas postadas pelo usuário $A$ e vistas por $B$ e $P_{BvA}$ o número de respostas postadas pelo usuário $B$ e vistas por $A$.

Aplicando a equação \ref{eq:k2}, o incremento do valor de $k$ para cada resposta curtida seria menor para o caso de muitas perguntas terem sido vistas e poucas respostas terem sido dadas. De maneira inversa, quanto menor o conjunto de perguntas trocadas entre os usuários, maior o peso que uma resposta curtida terá sobre $k$.

Seria conveniente também, afetar as perguntas postadas com uma validade por cronologia ou fixar um valor máximo para o total de perguntas postadas pelos usuários $A$ e $B$. Para a implementação do produto neste trabalho, não foi considerada a idade das perguntas nem estabelecido um valor limite do somatório de perguntas postadas por $A$ e $B$ pois seria inconveniente para o desenvolvimento.

%=====================================================



		% conclusão

%=====================================================

% Estilos de bibliografia recomendados (só descomentar um estilo!)
% Mais infos: https://pt.sharelatex.com/learn/Bibtex_bibliography_styles
\bibliographystyle{apalike-ptbr}	% [Maziero et al., 2006]
%\bibliographystyle{alpha}		% [Maz06]
%\bibliographystyle{plainnat}		% vide Google "LaTeX Natbib"
%\bibliographystyle{plain}		% [1] ordem alfabética
%\bibliographystyle{unsrt}		% [1] ordem de uso no texto

% no estilo "unsrt", evita que citações nos índices sejam consideradas
%\usepackage{notoccite}

% base de bibliografia (BibTeX)
\bibliography{referencias}
%\bibliography{file1, file2, file3} % se tiver mais de um arquivo BibTeX

%=====================================================

% inclusão de apêndices
\appendix

% inclusão de apêndice
% \chapter{Metodologia}
\label{cap:metodologia}

% figuras estão no subdiretório "figuras/" dentro deste capítulo
\graphicspath{\currfiledir/figuras/}

%=====================================================

\section{Método de desenvolvimento}

O produto foi desenvolvido pelo modelo de cascata. A priorização das atividades foi baseada na competência da equipe em cada área de desenvolvimento. Dessa maneira, o \emph{front end} foi considerado o foco e a prioridade no desenvolvimento, tendo em vista que o nível de conhecimento da equipe na área de design e usabilidade era perceptivelmente menor do que a habilidade para o desenvolvimento do \emph{back end}.

A análise de requisitos…

Um MVP, foi desenvolvido durante a fase de projeto. A figura \ref{fig:mvpCriarQuest} apresenta a tela de criação de perguntas do MVP. Neste produto, é possível criar perguntas (\ref{fig:mvpCriarQuest}), responder perguntas postadas por outros usuários (\ref{fig:mvpVerQuest}) e marcar essas respostas (\ref{fig:mvpVerResp}). Assim, o produto já tem informações suficientes para criar um grafo com os usuários no qual o peso das arestas é o nível de afinidade entre os eles baseado na quantidade de respostas marcadas.

\begin{figure}[!htb]
\centering
\includegraphics[width=14cm]{mvpCriarQuest.png}
\caption{Tela de criação de perguntas do MVP. Fonte: os autores.}
\label{fig:mvpCriarQuest}
\end{figure}

\begin{figure}[!htb]
\centering
\includegraphics[width=14cm]{mvpVerQuest.png}
\caption{Tela de visualização de perguntas do MVP. Nesta tela é possível escolher uma pergunta para ser respondida. Fonte: os autores.}
\label{fig:mvpVerQuest}
\end{figure}

\begin{figure}[!htb]
\centering
\includegraphics[width=14cm]{mvpVerResp.png}
\caption{Tela de visualização de respostas recebidas do MVP. Nesta tela é possível marcar perguntas favoritas. Fonte: os autores.}
\label{fig:mvpVerResp}
\end{figure}

\begin{figure}[!htb]
\centering
\includegraphics[width=14cm]{mvpVerGrafo.png}
\caption{Visualização do grafo que representa as ligações entre usuários da rede social no MVP. Fonte: os autores.}
\label{fig:mvpVerGrafo}
\end{figure}

\FloatBarrier


A figura \ref{fig:mvpVerGrafo} é a tela do MVP que representa os usuários e suas conexões criadas na rede social por meio de um grafo.


\section{Arquitetura}

\begin{itemize}
\item Qual a tecnologia
\item Custo
\item Arquitetura
\end{itemize}

Desenho da arquitetura

\begin{figure}[!htb]
\centering
\includegraphics[width=14cm]{arquitetura.png}
\caption{Arquitetura do software. Fonte: os autores.}
\label{fig:arquitetura}
\end{figure}

\FloatBarrier
%=====================================================

\section{Tecnologia aplicada}

\begin{itemize}
\item Linguagens de programação
\item Frameworks

\end{itemize}

Frameworks: IONIC django

angular e github.

Linguagens: python typescript javascript html/css
Hardware: PC Notebook thinkpad x201, SSD 120GB e 8GB de RAM primeira geracao do i5 e arch linux, server usa debian stretch
Infraestrutura: 	Docker
VPS: CPU: Intel (Haswell, noTSX) (1)@2.3GHz GPU: CirrusLogicGD5446 Memory: 583MB / 1956MB.



%=====================================================

\section{Análise do sistema}

Diagramas de caso de USO
\begin{figure}[!htb]
\centering
\includegraphics[width=16cm]{DCU1.png}
\caption{Diagrama de caso de uso nível 1. Fonte: os autores.}
\label{fig:DCU1}
\end{figure}

\begin{figure}[!htb]
\centering
\includegraphics[width=16cm]{DCU2.png}
\caption{Diagrama de caso de uso nível 2. Fonte: os autores.}
\label{fig:DCU2}
\end{figure}
%=====================================================

Diagramas de sequência

\begin{figure}[!htb]
\centering
\includegraphics[width=16cm]{UC001-VisualizarQuestao.png}
\caption{Diagrama de caso de uso UC001 - Visualizar Questão. Fonte: os autores.}
\label{fig:UC001}
\end{figure}

\begin{figure}[!htb]
\centering
\includegraphics[width=16cm]{UC002-ResponderQuestao.png}
\caption{Diagrama de caso de uso UC002 - Responder Questão. Fonte: os autores.}
\label{fig:UC002}
\end{figure}

\begin{figure}[!htb]
\centering
\includegraphics[width=16cm]{UC003-VisualizarRespostas.png}
\caption{Diagrama de caso de uso UC003 - Visualizar Respostas. Fonte: os autores.}
\label{fig:UC003}
\end{figure}


\begin{figure}[!htb]
\centering
\includegraphics[width=16cm]{UC004-CriarQuestao.png}
\caption{Diagrama de caso de uso UC004 - Criar Questão. Fonte: os autores.}
\label{fig:UC004}
\end{figure}


\begin{figure}[!htb]
\centering
\includegraphics[width=16cm]{UC005-VisualizarCombinacoes.png}
\caption{Diagrama de caso de uso UC005 - Visualizar Combinações. Fonte: os autores.}
\label{fig:UC005}
\end{figure}


Diagrama de Classes

\begin{figure}[!htb]
\centering
\includegraphics[width=16cm]{DiagramaClasse.png}
\caption{Diagrama de classes. Fonte: os autores.}
\label{fig:diagramaClasse}
\end{figure}
\FloatBarrier



%=====================================================

\section{Cálculo de do peso da aresta}
O peso de uma aresta do grafo, $k$, representa o nível de afinidade entre os dois usuários conectados por esta aresta. Um usuário pode estar conectado a vários outros usuários.

Quando um usuário entra pela primeira vez na rede social, ele é obrigado a preencher um questionário contendo as questões constantes na tabela \ref{tab:questoes}. A partir das respostas deste questionário, um valor para $k$ é calculado levando em conta, tão somente, a similaridade entre as respostas de cada usuário.

\begin{table}[!htp]
\centering
\caption{Formulário inicial}
\label{tab:questoes}
\begin{tabular}{ || c ||}
\hline
Pergunta\\
\hline
\hline
Você prefere cachorro ou gato?\\  
\hline
Você prefere rock ou funk?\\
\hline
Você prefere verão ou inverno? \\
\hline
Você prefere cinema ou teatro?\\
\hline
Você prefere cerveja ou vinho?\\
\hline
Você prefere o dia ou a noite?\\
\hline
Você prefere sair ou ficar em casa?\\
\hline
Você fuma?\\
\hline
Você tem alguma religião?\\
\hline
Você acredita em signos?\\
\hline
Você prefere praia ou campo?\\
\hline
\end{tabular}
\end{table}

Para a determinação inicial de $k$, logo após o preenchimento do formulário, é calculado o total de respostas iguais entre dois usuários e aplicada a seguinte equação:

\begin{equation}
k_{ab} = \frac{R_{ab}*(x-1)}{N_{p}}
 \label{eq:k0}
\end{equation}


Onde $R_{ab}$ é o total de respostas do usuário A iguais ao usuário B; $x$ é o valor objetivo para considerar dois usuários similares - este valor será discutido adiante nesta seção - e $N_{p}$ é o total de perguntas do questionário inicial.

Na equação \ref{eq:k0}, o numerador é multiplicado por $x-1$ para que os usuários que tiveram todas as questões respondidas da mesma maneira no questionário tão somente fiquem muito próximos da margem que define a habilitação do mensageiro instantâneo. O objetivo é tornar obrigatória a interação por meio de perguntas e respostas antes que dois usuários possam ser considerados similares o suficiente para desfrutarem do mensageiro.

Então, conforme as perguntas postadas pelos usuários vão sendo respondidas e apreciadas, valor do peso da aresta, definido como $k_{ab}$, que liga os usuários $A$ e $B$, é recalculado, para cada resposta apontada como apreciada, a partir da seguinte definição:

\begin{equation}
k_{ab} = k_{ab} + \frac{(R_{ab} + R_{ba})}{(P_{A} + P_{B})}
\label{eq:k1}
\end{equation}

Onde $R_{ab}$ é o número de respostas que o usuário $A$ recebeu do usuário $B$ e $A$ gostou; $R_{ba}$ é o número de respostas que o usuário $B$ recebeu do usuário $A$ e a $B$ gostou; $P_{A}$ é o número de perguntas postadas pelo usuário $A$ e $P_{B}$ o número de respostas postadas pelo usuário $B$.

Na figura \ref{fig:grafico_k1}, podemos ver a evolução do peso $k$ para cada resposta recebida pelo usuário $A$ que ele gostou. Nesta situação o crescimento do valor de $k$ é linear, pois todas as perguntas postadas pelo usuário $A$ estão respondidas por $B$ e as respectivas respostas foram curtidas.

\begin{figure}[!htb]
\centering
\includegraphics[width=14cm]{grafico_k1.png}
\caption{Evolução de $k$ se todas as perguntas postadas por $A$ forem respondidas por $B$. Fonte: os autores.}
\label{fig:grafico_k1}
\end{figure}

Como o cálculo leva em conta todas as perguntas postadas pelo usuário $A$, no caso de já haver perguntas postadas e não respondidas, o crescimento do valor de $k$ é afetado num grau inversamente proporcional ao número de perguntas postadas, como pode-se observar na figura \ref{fig:grafico_k2}.

\begin{figure}[!htb]
\centering
\includegraphics[width=14cm]{grafico_k2.png}
\caption{Evolução de $k$ se $A$ postou 50 perguntas e apenas 25 respostas de $B$ foram curtidas. Fonte: os autores.}
\label{fig:grafico_k2}
\end{figure}

É importante observar que o peso do incremento para $k$ é menor quando maior for o número de perguntas postadas seja por $A$ ou $B$. Dessa maneira, podemos depreender que a interação entre usuários novos, ou com poucas perguntas postadas, terá influência maior no crescimento do valor de $k$. Por outro lado, a interação entre usuários com muitas perguntas já postadas, incrementará um valor menor sobre $k$.

Tendo calculado um peso para cada aresta que liga os usuários, foi necessário definir um limiar $x$ que seria o valor mínimo de $k$ para considerar que dois usuários são similares o suficiente para serem postos em contato sob o mensageiro instantâneo.

A definição de $x$ provou-se um desafio, tendo em vista que o comportamento de $k$ varia em função do número total de perguntas postadas por dois usuários em análise. A figura \ref{fig:poucasquestoes} mostra que a convergência para $x = 1$ para o caso de de dois usuários que postaram poucas perguntas é rápida, pois o incremento de $k$ é maior. No caso simulado na figura \ref{fig:poucasquestões}, pouco mais de cinco respostas curtidas foram o suficiente para atingir o $x$.

\begin{figure}[!htb]
\centering
\includegraphics[width=14cm]{poucasquestoes.png}
\caption{Evolução de $k$ em função das respostas curtidas se $A$ e $B$ têm poucas perguntas postadas. Ex.: menos que 10 perguntas. Fonte: os autores.}
\label{fig:poucasquestoes}
\end{figure}

A figura \ref{fig:muitasquestoes}  demonstra que a convergência de $k$ e $x$ para usuários que já postaram muitas perguntas demanda muito mais respostas curtidas, neste caso mais que vinte respostas. Isto deve-se ao fato de ter sido decidido que a equação \ref{eq:k1} tem como divisor a soma de todas as perguntas já postadas pelos dois usuários, portanto, o incremento de $k$ é inversamente proporcional à soma de todas as perguntas postadas pelos usuários. 

\begin{figure}[!htb]
\centering
\includegraphics[width=14cm]{muitasquestoes.png}
\caption{Evolução de $k$ em função das respostas curtidas se $A$ e $B$ têm muitas perguntas postadas. Ex.: mais que 50 perguntas. Fonte: os autores.}
\label{fig:muitasquestoes}
\end{figure}

Uma adequação simplista para este problema seria definir $k$ como o número total de respostas curtidas entre dois usuários. Assim, $x$ seria o número mínimo de respostas que teriam que ser curtidas entre os usuários para que fossem considerados similares o suficiente para contatarem-se pelo mensageiro instantâneo. Porém, dessa maneira ignora-se que, se houverem muitas perguntas postadas por $A$ sem resposta de $B$ é possível que hajam mais diferenças do que similaridades entre os usuários, haja vista que muitas perguntas são ignoradas.

Portanto, é imprescindível considerar o volume de perguntas postadas. Uma evolução poderia adequar o software para contabilizar o número de perguntas postadas por $A$ e \emph{vistas} por $B$. Assim, poderíamos adequar o cálculo de $k$ como segue: 

\begin{equation}
k = k + \frac{R_{ab} + R_{ba}}{P_{AvB} + P_{BvA}} 
\label{eq:k2}
\end{equation}

Onde $R_{ab}$ é o número de respostas que o usuário $A$ recebeu do usuário $B$ e $A$ gostou; $R_{ba}$ é o número de respostas que o usuário $B$ recebeu do usuário $A$ e a $B$ gostou; $P_{AvB}$ é o número de perguntas postadas pelo usuário $A$ e vistas por $B$ e $P_{BvA}$ o número de respostas postadas pelo usuário $B$ e vistas por $A$.

Aplicando a equação \ref{eq:k2}, o incremento do valor de $k$ para cada resposta curtida seria menor para o caso de muitas perguntas terem sido vistas e poucas respostas terem sido dadas. De maneira inversa, quanto menor o conjunto de perguntas trocadas entre os usuários, maior o peso que uma resposta curtida terá sobre $k$.

Seria conveniente também, afetar as perguntas postadas com uma validade por cronologia ou fixar um valor máximo para o total de perguntas postadas pelos usuários $A$ e $B$. Para a implementação do produto neste trabalho, não foi considerada a idade das perguntas nem estabelecido um valor limite do somatório de perguntas postadas por $A$ e $B$ pois seria inconveniente para o desenvolvimento.

%=====================================================





% inclusão de apêndice
\chapter{Metodologia}
\label{cap:metodologia}

% figuras estão no subdiretório "figuras/" dentro deste capítulo
\graphicspath{\currfiledir/figuras/}

%=====================================================

\section{Método de desenvolvimento}

O produto foi desenvolvido pelo modelo de cascata. A priorização das atividades foi baseada na competência da equipe em cada área de desenvolvimento. Dessa maneira, o \emph{front end} foi considerado o foco e a prioridade no desenvolvimento, tendo em vista que o nível de conhecimento da equipe na área de design e usabilidade era perceptivelmente menor do que a habilidade para o desenvolvimento do \emph{back end}.

A análise de requisitos…

Um MVP, foi desenvolvido durante a fase de projeto. A figura \ref{fig:mvpCriarQuest} apresenta a tela de criação de perguntas do MVP. Neste produto, é possível criar perguntas (\ref{fig:mvpCriarQuest}), responder perguntas postadas por outros usuários (\ref{fig:mvpVerQuest}) e marcar essas respostas (\ref{fig:mvpVerResp}). Assim, o produto já tem informações suficientes para criar um grafo com os usuários no qual o peso das arestas é o nível de afinidade entre os eles baseado na quantidade de respostas marcadas.

\begin{figure}[!htb]
\centering
\includegraphics[width=14cm]{mvpCriarQuest.png}
\caption{Tela de criação de perguntas do MVP. Fonte: os autores.}
\label{fig:mvpCriarQuest}
\end{figure}

\begin{figure}[!htb]
\centering
\includegraphics[width=14cm]{mvpVerQuest.png}
\caption{Tela de visualização de perguntas do MVP. Nesta tela é possível escolher uma pergunta para ser respondida. Fonte: os autores.}
\label{fig:mvpVerQuest}
\end{figure}

\begin{figure}[!htb]
\centering
\includegraphics[width=14cm]{mvpVerResp.png}
\caption{Tela de visualização de respostas recebidas do MVP. Nesta tela é possível marcar perguntas favoritas. Fonte: os autores.}
\label{fig:mvpVerResp}
\end{figure}

\begin{figure}[!htb]
\centering
\includegraphics[width=14cm]{mvpVerGrafo.png}
\caption{Visualização do grafo que representa as ligações entre usuários da rede social no MVP. Fonte: os autores.}
\label{fig:mvpVerGrafo}
\end{figure}

\FloatBarrier


A figura \ref{fig:mvpVerGrafo} é a tela do MVP que representa os usuários e suas conexões criadas na rede social por meio de um grafo.


\section{Arquitetura}

\begin{itemize}
\item Qual a tecnologia
\item Custo
\item Arquitetura
\end{itemize}

Desenho da arquitetura

\begin{figure}[!htb]
\centering
\includegraphics[width=14cm]{arquitetura.png}
\caption{Arquitetura do software. Fonte: os autores.}
\label{fig:arquitetura}
\end{figure}

\FloatBarrier
%=====================================================

\section{Tecnologia aplicada}

\begin{itemize}
\item Linguagens de programação
\item Frameworks

\end{itemize}

Frameworks: IONIC django

angular e github.

Linguagens: python typescript javascript html/css
Hardware: PC Notebook thinkpad x201, SSD 120GB e 8GB de RAM primeira geracao do i5 e arch linux, server usa debian stretch
Infraestrutura: 	Docker
VPS: CPU: Intel (Haswell, noTSX) (1)@2.3GHz GPU: CirrusLogicGD5446 Memory: 583MB / 1956MB.



%=====================================================

\section{Análise do sistema}

Diagramas de caso de USO
\begin{figure}[!htb]
\centering
\includegraphics[width=16cm]{DCU1.png}
\caption{Diagrama de caso de uso nível 1. Fonte: os autores.}
\label{fig:DCU1}
\end{figure}

\begin{figure}[!htb]
\centering
\includegraphics[width=16cm]{DCU2.png}
\caption{Diagrama de caso de uso nível 2. Fonte: os autores.}
\label{fig:DCU2}
\end{figure}
%=====================================================

Diagramas de sequência

\begin{figure}[!htb]
\centering
\includegraphics[width=16cm]{UC001-VisualizarQuestao.png}
\caption{Diagrama de caso de uso UC001 - Visualizar Questão. Fonte: os autores.}
\label{fig:UC001}
\end{figure}

\begin{figure}[!htb]
\centering
\includegraphics[width=16cm]{UC002-ResponderQuestao.png}
\caption{Diagrama de caso de uso UC002 - Responder Questão. Fonte: os autores.}
\label{fig:UC002}
\end{figure}

\begin{figure}[!htb]
\centering
\includegraphics[width=16cm]{UC003-VisualizarRespostas.png}
\caption{Diagrama de caso de uso UC003 - Visualizar Respostas. Fonte: os autores.}
\label{fig:UC003}
\end{figure}


\begin{figure}[!htb]
\centering
\includegraphics[width=16cm]{UC004-CriarQuestao.png}
\caption{Diagrama de caso de uso UC004 - Criar Questão. Fonte: os autores.}
\label{fig:UC004}
\end{figure}


\begin{figure}[!htb]
\centering
\includegraphics[width=16cm]{UC005-VisualizarCombinacoes.png}
\caption{Diagrama de caso de uso UC005 - Visualizar Combinações. Fonte: os autores.}
\label{fig:UC005}
\end{figure}


Diagrama de Classes

\begin{figure}[!htb]
\centering
\includegraphics[width=16cm]{DiagramaClasse.png}
\caption{Diagrama de classes. Fonte: os autores.}
\label{fig:diagramaClasse}
\end{figure}
\FloatBarrier



%=====================================================

\section{Cálculo de do peso da aresta}
O peso de uma aresta do grafo, $k$, representa o nível de afinidade entre os dois usuários conectados por esta aresta. Um usuário pode estar conectado a vários outros usuários.

Quando um usuário entra pela primeira vez na rede social, ele é obrigado a preencher um questionário contendo as questões constantes na tabela \ref{tab:questoes}. A partir das respostas deste questionário, um valor para $k$ é calculado levando em conta, tão somente, a similaridade entre as respostas de cada usuário.

\begin{table}[!htp]
\centering
\caption{Formulário inicial}
\label{tab:questoes}
\begin{tabular}{ || c ||}
\hline
Pergunta\\
\hline
\hline
Você prefere cachorro ou gato?\\  
\hline
Você prefere rock ou funk?\\
\hline
Você prefere verão ou inverno? \\
\hline
Você prefere cinema ou teatro?\\
\hline
Você prefere cerveja ou vinho?\\
\hline
Você prefere o dia ou a noite?\\
\hline
Você prefere sair ou ficar em casa?\\
\hline
Você fuma?\\
\hline
Você tem alguma religião?\\
\hline
Você acredita em signos?\\
\hline
Você prefere praia ou campo?\\
\hline
\end{tabular}
\end{table}

Para a determinação inicial de $k$, logo após o preenchimento do formulário, é calculado o total de respostas iguais entre dois usuários e aplicada a seguinte equação:

\begin{equation}
k_{ab} = \frac{R_{ab}*(x-1)}{N_{p}}
 \label{eq:k0}
\end{equation}


Onde $R_{ab}$ é o total de respostas do usuário A iguais ao usuário B; $x$ é o valor objetivo para considerar dois usuários similares - este valor será discutido adiante nesta seção - e $N_{p}$ é o total de perguntas do questionário inicial.

Na equação \ref{eq:k0}, o numerador é multiplicado por $x-1$ para que os usuários que tiveram todas as questões respondidas da mesma maneira no questionário tão somente fiquem muito próximos da margem que define a habilitação do mensageiro instantâneo. O objetivo é tornar obrigatória a interação por meio de perguntas e respostas antes que dois usuários possam ser considerados similares o suficiente para desfrutarem do mensageiro.

Então, conforme as perguntas postadas pelos usuários vão sendo respondidas e apreciadas, valor do peso da aresta, definido como $k_{ab}$, que liga os usuários $A$ e $B$, é recalculado, para cada resposta apontada como apreciada, a partir da seguinte definição:

\begin{equation}
k_{ab} = k_{ab} + \frac{(R_{ab} + R_{ba})}{(P_{A} + P_{B})}
\label{eq:k1}
\end{equation}

Onde $R_{ab}$ é o número de respostas que o usuário $A$ recebeu do usuário $B$ e $A$ gostou; $R_{ba}$ é o número de respostas que o usuário $B$ recebeu do usuário $A$ e a $B$ gostou; $P_{A}$ é o número de perguntas postadas pelo usuário $A$ e $P_{B}$ o número de respostas postadas pelo usuário $B$.

Na figura \ref{fig:grafico_k1}, podemos ver a evolução do peso $k$ para cada resposta recebida pelo usuário $A$ que ele gostou. Nesta situação o crescimento do valor de $k$ é linear, pois todas as perguntas postadas pelo usuário $A$ estão respondidas por $B$ e as respectivas respostas foram curtidas.

\begin{figure}[!htb]
\centering
\includegraphics[width=14cm]{grafico_k1.png}
\caption{Evolução de $k$ se todas as perguntas postadas por $A$ forem respondidas por $B$. Fonte: os autores.}
\label{fig:grafico_k1}
\end{figure}

Como o cálculo leva em conta todas as perguntas postadas pelo usuário $A$, no caso de já haver perguntas postadas e não respondidas, o crescimento do valor de $k$ é afetado num grau inversamente proporcional ao número de perguntas postadas, como pode-se observar na figura \ref{fig:grafico_k2}.

\begin{figure}[!htb]
\centering
\includegraphics[width=14cm]{grafico_k2.png}
\caption{Evolução de $k$ se $A$ postou 50 perguntas e apenas 25 respostas de $B$ foram curtidas. Fonte: os autores.}
\label{fig:grafico_k2}
\end{figure}

É importante observar que o peso do incremento para $k$ é menor quando maior for o número de perguntas postadas seja por $A$ ou $B$. Dessa maneira, podemos depreender que a interação entre usuários novos, ou com poucas perguntas postadas, terá influência maior no crescimento do valor de $k$. Por outro lado, a interação entre usuários com muitas perguntas já postadas, incrementará um valor menor sobre $k$.

Tendo calculado um peso para cada aresta que liga os usuários, foi necessário definir um limiar $x$ que seria o valor mínimo de $k$ para considerar que dois usuários são similares o suficiente para serem postos em contato sob o mensageiro instantâneo.

A definição de $x$ provou-se um desafio, tendo em vista que o comportamento de $k$ varia em função do número total de perguntas postadas por dois usuários em análise. A figura \ref{fig:poucasquestoes} mostra que a convergência para $x = 1$ para o caso de de dois usuários que postaram poucas perguntas é rápida, pois o incremento de $k$ é maior. No caso simulado na figura \ref{fig:poucasquestões}, pouco mais de cinco respostas curtidas foram o suficiente para atingir o $x$.

\begin{figure}[!htb]
\centering
\includegraphics[width=14cm]{poucasquestoes.png}
\caption{Evolução de $k$ em função das respostas curtidas se $A$ e $B$ têm poucas perguntas postadas. Ex.: menos que 10 perguntas. Fonte: os autores.}
\label{fig:poucasquestoes}
\end{figure}

A figura \ref{fig:muitasquestoes}  demonstra que a convergência de $k$ e $x$ para usuários que já postaram muitas perguntas demanda muito mais respostas curtidas, neste caso mais que vinte respostas. Isto deve-se ao fato de ter sido decidido que a equação \ref{eq:k1} tem como divisor a soma de todas as perguntas já postadas pelos dois usuários, portanto, o incremento de $k$ é inversamente proporcional à soma de todas as perguntas postadas pelos usuários. 

\begin{figure}[!htb]
\centering
\includegraphics[width=14cm]{muitasquestoes.png}
\caption{Evolução de $k$ em função das respostas curtidas se $A$ e $B$ têm muitas perguntas postadas. Ex.: mais que 50 perguntas. Fonte: os autores.}
\label{fig:muitasquestoes}
\end{figure}

Uma adequação simplista para este problema seria definir $k$ como o número total de respostas curtidas entre dois usuários. Assim, $x$ seria o número mínimo de respostas que teriam que ser curtidas entre os usuários para que fossem considerados similares o suficiente para contatarem-se pelo mensageiro instantâneo. Porém, dessa maneira ignora-se que, se houverem muitas perguntas postadas por $A$ sem resposta de $B$ é possível que hajam mais diferenças do que similaridades entre os usuários, haja vista que muitas perguntas são ignoradas.

Portanto, é imprescindível considerar o volume de perguntas postadas. Uma evolução poderia adequar o software para contabilizar o número de perguntas postadas por $A$ e \emph{vistas} por $B$. Assim, poderíamos adequar o cálculo de $k$ como segue: 

\begin{equation}
k = k + \frac{R_{ab} + R_{ba}}{P_{AvB} + P_{BvA}} 
\label{eq:k2}
\end{equation}

Onde $R_{ab}$ é o número de respostas que o usuário $A$ recebeu do usuário $B$ e $A$ gostou; $R_{ba}$ é o número de respostas que o usuário $B$ recebeu do usuário $A$ e a $B$ gostou; $P_{AvB}$ é o número de perguntas postadas pelo usuário $A$ e vistas por $B$ e $P_{BvA}$ o número de respostas postadas pelo usuário $B$ e vistas por $A$.

Aplicando a equação \ref{eq:k2}, o incremento do valor de $k$ para cada resposta curtida seria menor para o caso de muitas perguntas terem sido vistas e poucas respostas terem sido dadas. De maneira inversa, quanto menor o conjunto de perguntas trocadas entre os usuários, maior o peso que uma resposta curtida terá sobre $k$.

Seria conveniente também, afetar as perguntas postadas com uma validade por cronologia ou fixar um valor máximo para o total de perguntas postadas pelos usuários $A$ e $B$. Para a implementação do produto neste trabalho, não foi considerada a idade das perguntas nem estabelecido um valor limite do somatório de perguntas postadas por $A$ e $B$ pois seria inconveniente para o desenvolvimento.

%=====================================================





%=====================================================

\end{document}

%=====================================================
