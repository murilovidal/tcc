\begin{resumo}
\singlespacing
\noindent
As relações sociais são cada vez mais realizadas em ambiente virtual, o que traz a possibilidade de encontrar pessoas de várias culturas e com diferentes opiniões. Porém, por vezes o usuário da rede social quer apenas encontrar seus pares e compartilhar interesses mútuos. Este trabalho visa construir uma rede social com foco em sugestão de novos relacionamentos baseados em interesses mútuos, sem priorizar a aparência do usuário. Nesta rede, os usuários podem relacionar-se anonimamente por uma dinâmica de perguntas e respostas, na qual é possível compartilhar questões, responder perguntas dos outros usuários e avaliar as respostas recebidas. Conforme a interação nesse modelo vai acontecendo, o sistema convida os contatos considerados similares para relacionarem-se mais proximamente pelo uso de um mensageiro instantâneo particular. Dessa maneira, os usuários entram em contato com pessoas novas tendo sido mensurada a similaridade entre cada usuário com base em um cálculo determinístico. A rede social tratada neste trabalho é representada por um grafo no qual os nós são os usuários, as arestas são as conexões entre eles e o peso das arestas determina o grau de similaridade entre os usuários. Este método de sugestão de contatos tem a ambição de ser eficaz, de maneira que demande pouca carga de processamento, pois, é uma função do número de respostas recebidas e avaliadas. Ao longo do desenvolvimento do software o cálculo do peso da aresta foi testado com várias equações e os resultados foram comparados  entre si. Uma equação foi selecionada como a mais adequada e os resultados foram considerados satisfatórios. O cálculo do valor do peso da aresta do grafo é eficiente e leve; o produto final é atrativo e a dinâmica de perguntas e respostas tem a possibilidade de direcionar o usuário para relacionamentos que tratem de assuntos que lhe interessem naturalmente. Outras possibilidades de pesquisas futuras foram iluminadas com este trabalho de maneira que surgiram novas ideias de incremento na sofisticação do cálculo do peso das arestas do grafo, seja pelo emprego de inteligência artificial ou pela mineração de dados provenientes das interações por meio da análise semântica dos textos compartilhados entre os usuários.

\end{resumo}
