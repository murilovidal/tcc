\begin{abstract}
\singlespacing
\noindent
The social relationships are increasingly taking place in the virtual environment, which brings the possibility of meeting people of various cultures and with divergent opinions. However, sometimes the user of the social media wishes only to meet his peers and share common interests. This work seeks to build a social media focused in suggesting new relationships based in common interests, without prioritizing the looks of the user. In this network, the users can relate anonymously by a dynamic of questions and answers, in which is possible to share questions, answer another users’ questions and evaluate the received answers. As the interaction in this model is happening, the system invites the contacts considered similar to relate closely by the use of a private instant messenger. That way, the users get in touch with new people as the similarity between each user is measured based in a deterministic calculation. The social media subject of this work is represented by a graph in which the nodes are the users, the edges are the connexion between them and the weight of the edges determines the grade of similarity between the users. This method of contacts suggestion has the ambition of being efficient, in a way that it demands little processing power as it is simply a function of the number of answers received and evaluated. Throughout the software development the edge weight calculation was tested with several equations and the results were compared to each other. An equation was selected as the most suitable and the results were considered satisfactory. The edge weight calculation is lightweight and efficient; the final product is attractive and the dynamic of questions and answers has the possibility of directing the user to relationships that  deal with subjects that interest them naturally. Another possibilities of further research were unveiled with this work as new ideas to increase the sophistication of the graph’s edge’s weight calculation, be it by employing artificial intelligence or by mining data from the interactions thru the semantic analysis of the texts shared between the users.

\end{abstract}
