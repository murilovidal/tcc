\chapter{Considerações Finais}
\label{cap:conclusao}

% figuras estão no subdiretório "figuras/" dentro deste capítulo
\graphicspath{\currfiledir/figuras/}

%=====================================================

As interações sociais em ambiente virtual são parte do dia a dia de muitas pessoas no mundo todo. Muitas vezes as interações em ambiente virtual são tão frequentes ou até mais do que os relacionamentos presenciais. Por isso podemos considerar relevante qualquer esforço de melhoria da experiência dos usuários das redes sociais.

Muitos algoritmos são desenvolvidos para otimizar a experiência do usuário na internet, e as redes sociais estão inclusas nesse esforço pois contém informação, publicidade, serviços e contatos importantes para  vida pessoal e profissional dos usuários.

Este trabalho buscou criar uma rede social na qual os usuários relacionam-se anonimamente por meio de troca de perguntas e respostas até que o sistema decida estabelecer o contato direto entre os usuários através de um mensageiro instantâneo. O desafio foi definir uma estrutura de dados e um método de cálculo de semelhança entre os usuários que fosse acuradamente agradável ao utilizador e demandasse de pouca carga de processamento.

Podemos afirmar que o objetivo foi atingido, uma vez que o software é completamente utilizável e apresenta a performance esperada. Os relacionamentos são criados a partir das interações iniciais e o grafo que representa a rede é atualizado corretamente a cada entrada de novas respostas.

A função que determina o valor da aresta que representa o nível de afinidade entre dois usuários é elegantemente simples e foi determinada como adequada após vários experimentos e descobertas ao longo do desenvolvimento do projeto. A solução simplificada garante bom desempenho e eficácia.

Além da satisfação de encontrar um bom resultado, é igualmente entusiasmante vislumbrar os novos caminhos que este sistema pode tomar para entregar seu valor com mais sofisticação. Pudemos observar que há vastas possibilidades de emprego de métodos de extração de dados a partir do léxico dos usuários com aplicação de inteligência artificial e que, a partir destes dados, algoritmos de recomendação ou de classificação mais complexos poderiam ser empregados para melhorar a experiência do usuário e a compreensão das interações estabelecidas nas redes sociais.

Evidente que a ampliação da nossa esfera de conhecimento percebida durante o desenvolvimento deste trabalho nos coloca em contato com o desconhecido. Este contato acelera o aprendizado e lança luz à novas possibilidades que podem ser exploradas em trabalhos futuros com o ímpeto criado por este.


%=====================================================
